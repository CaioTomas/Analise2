\documentclass[12pt,a4paper]{article}
\usepackage{amsmath,amssymb,amsthm}
\usepackage{makeidx,graphics}
\usepackage[dvips]{graphicx}
\usepackage[portuguese]{babel}
\usepackage[utf8]{inputenc}
\usepackage{ae}
\usepackage{indentfirst}
\usepackage{amsbsy}
\usepackage{fancyhdr}
\usepackage{pstricks}
\usepackage[all]{xy}
\usepackage{wrapfig}
\usepackage[pdfstartview=FitH,backref,colorlinks,bookmarksnumbered,bookmarksopen,linktocpage,urlcolor=blue,linkcolor=cyan]{hyperref}
\usepackage{bussproofs}
\usepackage{amsmath}
\usepackage{mathtools}
\usepackage{amsfonts}
\usepackage{wasysym}
\usepackage{url}
\usepackage{float} 
\usepackage{subcaption}
\usepackage{pgfplots}
\pgfplotsset{compat=newest}
\usepgfplotslibrary{fillbetween}
\usepackage[shortlabels]{enumitem}
\usepackage{esint}
\usepackage{multicol}
\usepackage{todonotes}

\newtheorem{definition}{Definição}
\newtheorem{lema}{Lema}
\newtheorem{teorema}{Teorema}
\newtheorem{corolario}{Corolário}
\newtheorem*{obs}{Observação}

\setlength{\topmargin}{-1.0in}
\setlength{\oddsidemargin}{0in}
\setlength{\evensidemargin}{0in}
\setlength{\textheight}{10.5in}
\setlength{\textwidth}{6.5in}
\setlength{\baselineskip}{12mm}

\newcommand{\dx}{\ \mathrm{d} x }
\newcommand{\dy}{\ \mathrm{d} y }
\newcommand{\dz}{\ \mathrm{d} z }
\newcommand{\du}{\ \mathrm{d} u }
\newcommand{\dv}{\ \mathrm{d} v }
\newcommand{\dr}{\ \mathrm{d} r }
\newcommand{\dt}{\ \mathrm{d} t }
\newcommand{\dteta}{\ \mathrm{d} \theta }
\newcommand{\dro}{\ \mathrm{d} \rho }
\newcommand{\dfi}{\ \mathrm{d} \phi }
\newcommand{\ds}{\ \mathrm{d} s }
\newcommand{\dS}{\ \mathrm{d} S }
\newcommand{\dq}{\ \mathrm{d} q }
\newcommand{\dif}{\mathrm{d}}
\newcommand{\res}{\mathrm{res}}

\DeclareMathOperator{\rot}{rot}
\DeclareMathOperator{\diverg}{div}
\DeclareMathOperator{\tr}{tr}
\DeclareMathOperator{\rank}{rank}
\DeclareMathOperator{\Sym}{Sym}

\newcommand{\C}{\mathbb{C}}
\newcommand{\R}{\mathbb{R}}
\newcommand{\Q}{\mathbb{Q}}
\newcommand{\Z}{\mathbb{Z}}
\newcommand{\N}{\mathbb{N}}
\renewcommand{\P}{\mathbb{P}}
\newcommand{\F}{\mathcal{F}}
\renewcommand{\Re}{\mathrm{Re}}
\renewcommand{\Im}{\mathrm{Im}}

\graphicspath{{img/}}

\renewcommand{\sectionmark}[1]{ \markright{ \thesection.\ #1}}

\title{\textbf{Análise 2}\\ Exercícios Cap. 4}
\author{Caio Tomás de Paula}
\date{28/02/2022}
\begin{document}

\maketitle

\paragraph{Exercício 1.}
    %
    \begin{proof}
        Tomando $h\in\R^m$, temos que
        %
        \begin{align*}
            \varphi(x+h) &= \left\langle T(x+h)\cdot f(x+h), g(x+h) \right\rangle \\
                         &= \left\langle [T(x) + T(h)]\cdot [f(x) + f'(x)\cdot h + \rho_f(h)], 
                                         g(x) + g'(x)\cdot h + \rho_g(h) \right\rangle.
        \end{align*}
        %
        Expandindo o produto interno usando a diferenciabilidade de $T, f$ e $g$, obtemos
        %
        \begin{align*}
            \varphi(x+h) &= \varphi(x) \\
                         &+ \left\langle T(x)\cdot f'(x)\cdot h, g(x) \right\rangle \\
                         &+ \left\langle T(x)\cdot f(x), g'(x)\cdot h \right\rangle \\
                         &+ \left\langle T(x)\cdot f'(x)\cdot h, g'(x)\cdot h \right\rangle \\
                         &+ \rho_{\varphi}(h),
        \end{align*}
        %
        com $\rho_{\varphi(h)}/h \xrightarrow{h\to\vec{0}} \vec{0}$. Logo, 
        %
        \begin{align*}
            \varphi'(x)\cdot h &= \left\langle T(x)\cdot f'(x)\cdot h, g(x) \right\rangle \\
                         &+ \left\langle T(x)\cdot f(x), g'(x)\cdot h \right\rangle \\
                         &+ \left\langle T(x)\cdot f'(x)\cdot h, g'(x)\cdot h \right\rangle.
        \end{align*}
        %
    \end{proof}
    %
\paragraph{Exercício 2.}
    %
    \begin{proof}
        Pela regra da cadeia,
        %
        \begin{equation*}
            (g\circ f)'(x) = Dg(f(x))\cdot f'(x).
        \end{equation*}
        %
        Diferenciando essa igualdade em $x$ e usando a regra da cadeia e a ``regra do produto'', temos
        %
        \begin{align*}
            (g\circ f)''(x) &= (Dg(f(x)))'\cdot f'(x) + Dg(f(x))\cdot f''(x) \\
                            &= D^2g(f(x))\cdot f'(x)\cdot f'(x) + Dg(f(x))\cdot f''(x) \\
                            &= g''(y)\cdot f'(x)\cdot f'(x) + g'(y)\cdot f''(x),
        \end{align*}
        %
        sendo $y=f(x)$.
    \end{proof}
    %
\paragraph{Exercício 3.}
%
    \begin{enumerate}[a)]
        \item Seja $S = f^{-1}(k)$ a superfície de nível de $f$ tal que $\lambda(J)\subset S$. Temos
        então que dado $t\in J$, $\lambda(t) = f^{-1}(c)$ para algum $c\in\R$ e, daí, $(f\circ\lambda)(t) = c$.
        Portanto, temos, por um lado, que
        %
        \begin{equation*}
            (f\circ \lambda)'(t) = 0,
        \end{equation*}
        %
        mas, usando a regra da cadeia, também temos que
        %
        \begin{equation*}
            0 = (f\circ \lambda)'(t) = Df(\lambda(t))\cdot\lambda'(t) 
                                     = \left\langle u(\lambda(t)), \lambda'(t) \right\rangle,
        \end{equation*}
        %
        ou seja, o gradiente de $f$ em $\lambda(t)$ é ortogonal a $\lambda'(t)$ para todo $t\in J$.
        %
        \item Para que $\partial f/\partial h (x)$ seja máximo, devemos ter $h$ na mesma direção do
        gradiente devido ao fato de que $\left\langle u,h \right\rangle = \partial f/\partial h (x)$,
        ou seja, $h = \alpha u$ com $\alpha\in\R$ positivo. Portanto, como $h$ tem norma 1, segue que
        %
        \begin{equation*}
            \|\alpha u\| = 1 \iff \alpha = \frac{1}{\|u\|}.
        \end{equation*}
        %
        Daí, segue que o valor máximo da derivada direcional é
        %
        \begin{equation*}
            \left\langle u, \alpha u \right\rangle = \alpha \|u\|^2 = \|u\|.
        \end{equation*}
        %
    \end{enumerate}
%
\paragraph{Exercício 4.}
    %
    \begin{proof}
        No caso de $f(tx) = tf(x)$, temos que
        %
        \begin{equation*}
            \frac{\partial f}{\partial v}(0) = \lim_{t\to 0} \frac{f(tv)}{t} = f(v),
        \end{equation*}
        %
        ou seja, $f'(0)\cdot v = f(v)$ e, portanto, $f$ é linear. Similarmente,
        no caso de $f(tx) = t^2f(x)$ temos que $f''(0)\cdot v = f(v)$, e podemos
        definir a função bilinear
        %
        \begin{equation*}
            B:\R^m\times\R^m \to \R^n
        \end{equation*}
        %
        por
        %
        \begin{equation*}
            B(x,y) = \frac{1}{2}\left[ f(x+y) - f(x) - f(y) \right],
        \end{equation*}
        %
        de modo que
        %
        \begin{equation*}
            B(x,x) = \frac{1}{2}\left[ f(2x) - 2f(x) \right]
                   = \frac{1}{2}[ 4f(x) - 2f(x) ]
                   = f(x), \forall x\in\R^m.
        \end{equation*}
        %
    \end{proof}
    %
    Quanto à observação, $f:\R^2\to\R$ dada por $f(x,y) = \sqrt{x^4 + y^4}$ satisfaz
    $f(tx,ty) = t^2f(x,y)$ mas não é quadrática porque apesar de existir
    %
    \begin{equation*}
        f'(x,y) = 2\left( \frac{x^3}{\sqrt{x^4 + y^4}}, \frac{y^3}{\sqrt{x^4 + y^4}} \right) 
                = 2(f'_1, f'_2),
    \end{equation*}
    %
    as funções coordenadas de $f'$ não são diferenciáveis na origem já que
    %
    \begin{align*}
        \partial_x f'_1 (0,0) &= \lim_{h\to 0} \frac{h^3/h^2 - 0}{h} = 1, \\
        \partial_y f'_1 (0,0) &= \lim_{h\to 0} \frac{0 - 0}{h} = 0
    \end{align*}
    %
    e
    %
    \begin{align*}
        \partial_x f'_2 (0,0) &= \lim_{h\to 0} \frac{0 - 0}{h} = 0, \\
        \partial_y f'_2 (0,0) &= \lim_{h\to 0} \frac{h^3/h^2 - 0}{h} = 1.
    \end{align*}
    %
\paragraph{Exercício 5.}
    %
    \begin{proof}
        Dado $h\in\R^m$, temos
        %
        \begin{align*}
            \varphi(x+h) &= f'(x+h)\cdot k \\
                         &= f'(x)\cdot k + (f''(x)\cdot h)\cdot k + \rho_{f'}(h)\cdot k \\
                         &= \varphi(x) + (f''(x)\cdot h)\cdot k + \rho_{f'}(h)\cdot k.
        \end{align*}
        %
        Como $\rho_{f'}(h)\cdot k \xrightarrow{h\to \vec{0}} \vec{0}$, segue que $(\varphi'(x)\cdot h)/h 
        = (f''(x)\cdot h)\cdot k$.
    \end{proof}
    %
\paragraph{Exercício 6.}
    %
    Temos que
    %
    \begin{align*}
        \varphi(t+h) = A(t(x+h)) &= A(tx) + A'(tx)\cdot h + \rho_A(h) \\
                                 &= \varphi(t) + A'(tx)\cdot h + \rho_A(h),
    \end{align*}
    %
    de modo que $\varphi'(t)\cdot h = (A'(tx)\cdot h)\cdot x$.
\paragraph{Exercício 7.}
    %
    \begin{proof}
        Vamos primeiro mostrar que as $i$-ésimas derivadas são $(k-i)$-homogêneas, com $0\leq i < k$.
        De fato, derivando em relação a $x$ e usando a regra da cadeia $i$ vezes, temos que
        %
        \begin{equation}
        \label{eq:homogeneidade-derivada}
            t^i f^{(i)}(tx) = t^kf^{(i)}(x) \implies f^{(i)}(tx) = t^{k-i}f^{(i)}(x).
        \end{equation}
        %
        Se derivarmos $k$ vezes, temos
        %
        \begin{equation*}
            t^k f^{(k)}(tx) = t^kf^{(k)}(x) \implies f^{(k)}(tx) = f^{(k)}(x),
        \end{equation*}
        %
        para todo $t\in\R$ e para todo $x\in\R^m$. Daí, segue que $f^{(k)}(0) = f^{(k)}(x)$ para todo 
        $x$, ou seja, $f^{(k)}$ é constante.
        
        Agora, derivando \eqref{eq:homogeneidade-derivada} $k-i$ vezes em relação a $t$, obtemos
        %
        \begin{align*}
            x^{k-i}\frac{\partial^{k-i}f^{(i)}}{\partial t^{k-i}}(tx) 
            = (k-i)!f^{(i)}(x).
        \end{align*}
        %
        Tomando $t=0$, segue que
        %
        \begin{align*}
            f^{(i)}(x) =
            \frac{x^{k-i}}{(k-i)!}\cdot\frac{\partial^{k-i}f^{(i)}}{\partial t^{k-i}}(0),
        \end{align*}
        %
        ou seja,
        %
        \begin{align*}
            f^{(i)}(x) =
            \frac{f^{(k)}(0)}{(k-i)!}\cdot x^{k-i}.
        \end{align*}
        %
        Daí, aplicando o resultado para $f:\mathcal{L}(\R^m)\to\mathcal{L}(\R^m)$ dada por $f(X) = X^k$,
        temos
        %
        \begin{equation*}
            f^{(i)}(X) = \frac{k!}{(k-1)!}X^{k-i} \implies 
            |f^{(i)}(X)| \leq \frac{k!}{(k-1)!}|X|^{k-i}.
        \end{equation*}
        %
    \end{proof}
    %
\paragraph{Exercício 8.}
    %
    \begin{proof}
        Temos que dado $v = (a,b)\in\R^2$,
        %
        \begin{align*}
            |f(v)| = |T| &= \sup_{|(x,y)| = 1} |T\cdot (x,y)| \\
                         &= \sup_{|(x,y)| = 1} |(ax, by)| \\
                         &= \sup_{|(x,y)| = 1} \max\{ |a|\cdot|x|, |b|\cdot |y| \} \\
                         &= \max\{ |a|, |b| \} \\
                         &= |v|.
        \end{align*}
        %
        Portanto, a norma de $F$ não é diferenciável, pois a norma do máximo não o é nas 
        diagonais $x = \pm y$. Generalizando, no espaço das matrizes reais $n\times m$, 
        a norma $|T| = \sup\{ |T\cdot v| : v\in\R^m, |v| = 1 \}$ não é diferenciável,
        pois $f(v) = T$ com $T\cdot (x_1, \dots, x_m) = (v_1x_1, \dots, v_mx_m)$ é tal
        que
        %
        \begin{align*}
            |f(v)| = |T| &= \sup_{|(x_1, \dots, x_m)| = 1} |T\cdot (x_1, \dots, x_m)| \\
                         &= \sup_{|(x_1, \dots, x_m)| = 1} |(v_1x_1, \dots, v_mx_m)| \\
                         &= \sup_{|(x_1, \dots, x_m)| = 1} \max\{ |v_1|\cdot|x_1|, \dots, 
                         |v_m|\cdot|x_m| \} \\
                         &= \max\{ |v_1|, \dots, |v_m| \} \\
                         &= |v|,
        \end{align*}
        %
        e a norma $\|\cdot\|_{\infty}$ em $\R^m$ não é diferenciável.
    \end{proof}
    %
\paragraph{Exercício 9.}
    %
    \begin{proof}
        Primeiro, observe que a adjunta de uma matriz com entradas reais nada mais
        é do que a transposta, haja vista que
        %
        \begin{align*}
            \left\langle A\cdot e_i^{m}, e_j^{n} \right\rangle
            &= j\text{-ésimo elemento da } i \text{-ésima coluna de } A \\
            = \left\langle e_i^{m}, A^*\cdot e_j^{n} \right\rangle
            &= i\text{-ésimo elemento da } j \text{-ésima coluna de } A^*,
        \end{align*}
        %
        onde os sobrescritos indicam a dimensão dos vetores das bases canônicas de $\R^m$ e $\R^n$.
        Com isso, definindo
        %
        \begin{equation*}
            \left\langle A,B \right\rangle = \tr(A^tB),
        \end{equation*}
        %
        temos que, para toda matriz $A$ real $n\times m$,
        %
        \begin{align*}
            \left\langle A,A \right\rangle &= \tr(A^tA) \\
                                           &= \sum_{i=1}^m (A^tA)_{ii} \\
                                           &= \sum_{i=1}^{m}\sum_{j=1}^{n} a^t_{ij}a_{ji} \\
                                           &= \sum_{i=1}^{m}\sum_{j=1}^{n} a^2_{ji} \geq 0
        \end{align*}
        %
        e também
        %
        \begin{align*}
            \left\langle A,A \right\rangle = \sum_{i=1}^{n}\sum_{j=1}^{m} a^2_{ji} = 0
            \iff a_{ji} = 0, \forall i,j \iff A = 0.
        \end{align*}
        %
        Ademais, como $\tr(A^t) = \tr(A)$, $\tr(A+B) = \tr(A)+\tr(B)$ e 
        $\tr(\lambda A) = \lambda\tr(A)$, segue que dadas matrizes $A,B$ e $C$ $n\times m$ e
        para qualquer $\lambda\in\R$, temos que
        %
        \begin{align*}
            \left\langle A,B \right\rangle = \tr(A^tB) 
            = \tr((B^tA)^t)
            = \tr(B^tA)
            = \left\langle B,A \right\rangle,
        \end{align*}
        %
        e
        %
        \begin{align*}
            \left\langle \lambda A + B, C \right\rangle 
            &= \tr( (\lambda A + B)^tC ) \\
            &= \tr( (\lambda A^t + B^t)C ) \\
            &= \lambda\tr(A^tC) + \tr(B^tC) \\
            &= \lambda\left\langle A,C \right\rangle + \left\langle B,C \right\rangle,
        \end{align*}
        %
        e segue que $\tr(A^tB)$ define um produto interno no espaço das matrizes reais $n\times m$.
        Consequentemente,
        %
        \begin{equation*}
            \|A\| = \sqrt{\tr(A^tA)} = \sqrt{ \sum_{i=1}^{m}\sum_{j=1}^{n} a^2_{ji} }
        \end{equation*}
        %
        define uma norma nesse mesmo espaço (diferenciável exceto no $0$?).
        
        Por fim, para todo $v\in\R^m$, temos que
        %
        \begin{align*}
            A\cdot v = f_1(v)\cdot e_1 + \cdots + f_n(v)\cdot e_n,
        \end{align*}
        %
        com $f_j:\R^m\to\R$ o funcional linear tal que $f_j(v) = \vec{a}_j\cdot v$, sendo
        $\vec{a}_j$ a $j$-ésima linha da matriz de $A$. Com essa notação, temos que
        %
        \begin{equation*}
            \|A\| = \sqrt{ |\vec{a}_1|^2 + \cdots + |\vec{a}_n|^2 }.
        \end{equation*}
        %
        Daí, basta notar que
        %
        \begin{align*}
            |A\cdot v|^2 &= |(\vec{a}_1\cdot v)e_1 + \cdots + (\vec{a}_n\cdot v)e_n|^2 \\
                         &= |\vec{a}_1\cdot v|^2 + \cdots + |\vec{a}_n\cdot v|^2 \\
                         &\leq (|\vec{a}_1|^2 + \cdots + |\vec{a}_n|^2)|v|^2 \\
                         &= \|A\|^2 \cdot |v|^2
        \end{align*}
        %
        pela ortonormalidade dos $e_j$'s e pela desigualdade de Cauchy-Schwarz. Tomando a raiz
        de ambos os lados, concluímos que
        %
        \begin{equation*}
            |A\cdot v| \leq \|A\|\cdot |v|,
        \end{equation*}
        %
        como desejado.
    \end{proof}
    %
\paragraph{Exercício 10.}
    %
    Pela regra da cadeia,
    %
    \begin{equation*}
        D(g\circ f)(x_0) = g'(f(x_0))\cdot f'(x_0) = g'(f(x_0))\cdot 0 = 0
    \end{equation*}
    %
    e, pelo Exercício 2, temos
    %
    \begin{equation*}
        D^2(g\circ f)(x_0) = g''(f(x_0))\cdot f'(x_0)\cdot f'(x_0) + g'(f(x))\cdot f''(x_0) = 0.
    \end{equation*}
    %
    Se derivarmos novamente, cada uma das parcelas terá a primeira, segunda ou terceira derivadas 
    de $f$ em $x_0$, de modo que cada parcela será $0$. Continuando a derivar até a ordem $k$,
    vemos que cada umas das parcelas terá pelo menos uma derivada de $f$ em $x_0$ e, portanto,
    $D^j(g\circ f)(x_0) = 0$ para $j=1,\dots,k$.
    
    Analogamente, se $f:U\to\R^n$ é $k$ vezes diferenciável e $D^jf(x_1) = 0$ para $j=1,\dots,k$
    então $D^j(f\circ h)(x_0) = 0$ para os mesmos valores de $j$, seja qual for $h:V\to\R^q$
    $k$vezes diferenciável com $h(V) \subset U$ e $h(x_0) = x_1$.
\paragraph{Exercício 11.}
    %
    \begin{proof}
        Suponha $f$ par. Então $f(x) = f(-x) = f(z)$, sendo $z=-x$, e temos
        %
        \begin{equation*}
            f'(x) = f'(z)\frac{dz}{dx} = -f'(-x) \implies f'(-x) = -f'(x),
        \end{equation*}
        %
        ou seja, a primeira derivada de uma função par é ímpar. Ademais, se $f$ é ímpar,
        temos $-f(x) = f(-x) = f(z)$ e 
        %
        \begin{equation*}
            -f'(x) = f'(z)\frac{dz}{dx} = -f'(-x) \implies f'(-x) = f'(x),
        \end{equation*}
        %
        ou seja, a primeira derivada de uma função ímpar é par.
        
        Portanto, se $f$ é par então sua primeira derivada é ímpar, a segunda é par, a terceira
        é ímpar, a quarta é par e assim por diante. Dito de outro modo, as derivadas de ordem par
        de uma função par são pares e as de ordem ímpar são ímpares. Além disso, para toda função
        ímpar $g$ temos que $g(0) = 0$ pois $g(0) = -g(0)$. Logo, $f^{(k)}(0) = 0$ para todo $k$
        ímpar.
        
        De semelhante modo, se $f$ é ímpar sua primeira derivada é par, a segunda é ímpar, a terceira
        é par, a quarta é ímpar e assim por diante, ou seja, as derivadas de ordem ímpar de uma função
        ímpar são pares e as de ordem par são ímpares. Ademais, $f^{(k)}(0) = 0$ para todo $k$ par,
        pelo mesmo motivo acima.
    \end{proof}
    %
\end{document}