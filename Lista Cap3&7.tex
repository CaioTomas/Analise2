\documentclass[12pt,a4paper]{article}
\usepackage{amsmath,amssymb,amsthm}
\usepackage{makeidx,graphics}
\usepackage[dvips]{graphicx}
\usepackage[portuguese]{babel}
\usepackage[utf8]{inputenc}
\usepackage{ae}
\usepackage{indentfirst}
\usepackage{amsbsy}
\usepackage{fancyhdr}
\usepackage{pstricks}
\usepackage[all]{xy}
\usepackage{wrapfig}
\usepackage[pdfstartview=FitH,backref,colorlinks,bookmarksnumbered,bookmarksopen,linktocpage,urlcolor=blue,linkcolor=cyan]{hyperref}
\usepackage{bussproofs}
\usepackage{amsmath}
\usepackage{mathtools}
\usepackage{amsfonts}
\usepackage{wasysym}
\usepackage{url}
\usepackage{float} 
\usepackage{subcaption}
\usepackage{pgfplots}
\pgfplotsset{compat=newest}
\usepgfplotslibrary{fillbetween}
\usepackage[shortlabels]{enumitem}
\usepackage{esint}
\usepackage{multicol}
\usepackage{todonotes}

\newtheorem{definition}{Definição}
\newtheorem{lema}{Lema}
\newtheorem{teorema}{Teorema}
\newtheorem{corolario}{Corolário}
\newtheorem*{obs}{Observação}

\setlength{\topmargin}{-1.0in}
\setlength{\oddsidemargin}{0in}
\setlength{\evensidemargin}{0in}
\setlength{\textheight}{10.5in}
\setlength{\textwidth}{6.5in}
\setlength{\baselineskip}{12mm}

\newcommand{\dx}{\ \mathrm{d} x }
\newcommand{\dy}{\ \mathrm{d} y }
\newcommand{\dz}{\ \mathrm{d} z }
\newcommand{\du}{\ \mathrm{d} u }
\newcommand{\dv}{\ \mathrm{d} v }
\newcommand{\dr}{\ \mathrm{d} r }
\newcommand{\dt}{\ \mathrm{d} t }
\newcommand{\dteta}{\ \mathrm{d} \theta }
\newcommand{\dro}{\ \mathrm{d} \rho }
\newcommand{\dfi}{\ \mathrm{d} \phi }
\newcommand{\ds}{\ \mathrm{d} s }
\newcommand{\dS}{\ \mathrm{d} S }
\newcommand{\dq}{\ \mathrm{d} q }
\newcommand{\dif}{\mathrm{d}}
\newcommand{\res}{\mathrm{res}}

\DeclareMathOperator{\rot}{rot}
\DeclareMathOperator{\diverg}{div}
\DeclareMathOperator{\tr}{tr}
\DeclareMathOperator{\rank}{rank}
\DeclareMathOperator{\Sym}{Sym}

\newcommand{\C}{\mathbb{C}}
\newcommand{\R}{\mathbb{R}}
\newcommand{\Q}{\mathbb{Q}}
\newcommand{\Z}{\mathbb{Z}}
\newcommand{\N}{\mathbb{N}}
\renewcommand{\P}{\mathbb{P}}
\newcommand{\F}{\mathcal{F}}
\renewcommand{\Re}{\mathrm{Re}}
\renewcommand{\Im}{\mathrm{Im}}

\graphicspath{{img/}}

\renewcommand{\sectionmark}[1]{ \markright{ \thesection.\ #1}}

\title{\textbf{Análise 2}\\ Exercícios Cap. 3 \& 7}
\author{Caio Tomás de Paula}
\date{23/03/2022}
\begin{document}

\maketitle

\section{Capítulo 3}

\paragraph{Exercício 1.}
%
\begin{proof}
    Sabemos, do livro-texto, que
    %
    \begin{equation*}
        f''(x) = \sum_{i=1}^m\sum_{j=1}^m \frac{ \partial^2 f }
        {\partial x_i \partial x_j} (x) \, dx_idx_j,
    \end{equation*}
    %
    onde $x_k$ denota a $k$-ésima entrada de $x$. Daí, segue que
    %
    \begin{equation*}
        f'''(x)\cdot e_i = 
        \frac{ \partial f'' }{ \partial x_i } (x)
        = \sum_{k=1}^m \frac{ \partial^3 f }
        {\partial x_i \partial x_j \partial x_k} (x) \, dx_k, \, i=1,\dots,m.
    \end{equation*}
    %
    Portanto,
    %
    \begin{equation*}
        f'''(x) = \sum_{k=1}^m\sum_{i=1}^m\sum_{j=1}^m 
        \frac{ \partial^3 f }
        {\partial x_i \partial x_j \partial x_k} (x) \, dx_idx_jdx_k,
    \end{equation*}
    %
    como desejado.
\end{proof}
%
\paragraph{Exercício 2.}
%
Defina
%
\begin{equation*}
    \varphi_i : 
    E_1 \times \cdots \times E_{i-1} \times E_{i+1} \times \cdots \times E_p \to
    \mathcal{L}(E,F)
\end{equation*}
%
por
%
\begin{equation*}
    \varphi_i(x_1, \dots, x_{i-1}, x_{i+1}, \dots, x_p) = 
    \varphi(x_1, \dots, x_{i-1}, \cdot, x_{i+1}, \dots, x_p).
\end{equation*}
%
Daí, usando a derivada de uma transformação $p$-linear, temos
%
\begin{align*}
    \varphi'(x_1, \dots, x_p)\cdot(v_1, \dots, v_p) &=
    \varphi(v_1, x_2, \dots, x_p) + \cdots + \varphi(x_1, x_2, \dots, v_p) \\
    &= (\varphi_1\circ\pi_1)(v_1, x_2, \dots, x_p) + \cdots +
    (\varphi_p\circ\pi_p)(x_1, x_2, \dots, v_p),
\end{align*}
%
ou seja,
%
\begin{equation*}
    \varphi' = \sum_{i=1}^p \varphi_i\circ\pi_i.
\end{equation*}
%
Como $\varphi'$ é a soma de funções $(p-1)$-lineares, segue que $\varphi''$
é a soma de funções $(p-2)$-lineares e assim por diante, ou seja, $\varphi^{(j)} = 0$
para $j\geq p+1$ e $\varphi\in C^{\infty}$.
%
\paragraph{Exercício 3.}
%
\begin{proof}
    Temos que
    %
    \begin{equation*}
        [f''(x,y,z)] = 
        \begin{bmatrix}
            6 & 0 & 2 \\
            0 & 4 & 0 \\
            2 & 0 & 2
        \end{bmatrix}.
    \end{equation*}
    %
    Daí, segue que
    %
    \begin{equation*}
        f''(x,y,z)\cdot (h,k) = 
        \begin{bmatrix}
            k_1 & k_2 & k_3
        \end{bmatrix}\cdot
        \begin{bmatrix}
            6 & 0 & 2 \\
            0 & 4 & 0 \\
            2 & 0 & 2
        \end{bmatrix}\cdot
        \begin{bmatrix}
            h_1 \\
            h_2 \\
            h_3
        \end{bmatrix} =
        6h_1k_1 + 2h_3k_1 + 4h_2k_2 + 2h_1k_3 + 2h_3k_3 
    \end{equation*}
    %
    e
    %
    \begin{equation*}
        f''(x,y,z) \cdot (h,h) = 6h_1^2 + 4h_2^2 + 2h_3^2 + 4h_1h_3
                               = 2h_1^2 + (2h_1 + h_3)^2 + h_3^2 + 4h_2^2 > 0
    \end{equation*}
    %
    sempre que $h = (h_1, h_2, h_3) \neq 0$.
\end{proof}
%
\paragraph{Exercício 4.}
%
Note que $f(x,y) = e^{2x}\cos(2y)$. Daí,
%
\begin{equation*}
    [f'(x,y)] =
    \begin{bmatrix}
        2e^{2x}\cos(2y) & -2e^{2x}\sin(2y)
    \end{bmatrix}
\end{equation*}
%
e
%
\begin{equation*}
    [f''(x,y)] =
    \begin{bmatrix}
        4e^{2x}\cos(2y) & -4e^{2x}\sin(2y) \\
        -4e^{2x}\sin(2y) & -4e^{2x}\cos(2y)
    \end{bmatrix} \implies
    [f''(0,0)] =
    \begin{bmatrix}
        4 & 0 \\
        0 & -4
    \end{bmatrix} = [A].
\end{equation*}
%
Portanto, 
%
\begin{equation*}
    A(w,w) =
    \begin{bmatrix}
        w_1 & w_2
    \end{bmatrix}\cdot
    \begin{bmatrix}
        4 & 0 \\
        0 & -4
    \end{bmatrix}\cdot
    \begin{bmatrix}
        w_1 \\
        w_2
    \end{bmatrix} =
    4w_1^2 - 4w_2^2,
\end{equation*}
%
para todo $w\in\R^2$. Logo, podemos tomar $u = (2,1)$ e $v = (1,2)$ para que $A(u,u) > 0$ e $A(v,v) < 0$.
%
\paragraph{Exercício 5.}\todo{conferir}
%
Temos que
%
\begin{equation*}
    [f'(x,y,z)] = 
    \begin{bmatrix}
        yz & xz & xy
    \end{bmatrix},
\end{equation*}
%
\begin{equation*}
    [f''(x,y,z)] =
    \begin{bmatrix}
        0 & z & y \\
        z & 0 & x \\
        y & x & 0
    \end{bmatrix}
\end{equation*}
%
e também
%
\begin{equation*}
    [f'''(x,y,z)] = 
    \begin{bmatrix}
        0 & 0 & 0 \\
        0 & 0 & 1 \\
        0 & 1 & 0 \\
        0 & 0 & 1 \\
        0 & 0 & 0 \\
        1 & 0 & 0 \\
        0 & 1 & 0 \\
        1 & 0 & 0 \\
        0 & 0 & 0
    \end{bmatrix}.
\end{equation*}
%
\paragraph{Exercício 6.}\todo{fazer}
%
\begin{enumerate}[a)]
    \item 
    \item
    \item
\end{enumerate}
%
\paragraph{Exercício 7.}
%
\begin{proof}
    Vamos proceder por indução em $j$. Se $j=1$, temos pela regra da cadeia que
    %
    \begin{equation*}
        (T\circ f)' = T'(f)\circ f' = T\circ f',
    \end{equation*}
    %
    já que $T$ é linear. Suponha, por hipótese de indução, que
    %
    \begin{equation*}
        (T\circ f)^{(k)} = T\circ f^{(k)}
    \end{equation*}
    %
    para um $k>1$. Então
    %
    \begin{equation*}
        (T\circ f)^{(k+1)} = (T\circ f^{(k)})' = T'(f^{(k)})\circ f^{(k+1)} = T\circ f^{(k+1)}
    \end{equation*}
    %
\end{proof}
%
\paragraph{Exercício 8.}\todo{fazer}
%
a
%
\section{Capítulo 7}

\paragraph{Exercício 1.} 
%

%
\paragraph{Exercício 2.}
%
\begin{enumerate}[a)]
    \item 
    \item
\end{enumerate}
%
\paragraph{Exercício 3.}
%

%
\paragraph{Exercício 4.} 
%

%
\paragraph{Exercício 5.} 
%

%
\paragraph{Exercício 6.}
%

%
\end{document}