\documentclass[12pt,a4paper]{article}
\usepackage{amsmath,amssymb,amsthm}
\usepackage{makeidx,graphics}
\usepackage[dvips]{graphicx}
\usepackage[portuguese]{babel}
\usepackage[utf8]{inputenc}
\usepackage{ae}
\usepackage{indentfirst}
\usepackage{amsbsy}
\usepackage{fancyhdr}
\usepackage{pstricks}
\usepackage[all]{xy}
\usepackage{wrapfig}
\usepackage[pdfstartview=FitH,backref,colorlinks,bookmarksnumbered,bookmarksopen,linktocpage,urlcolor=blue,linkcolor=cyan]{hyperref}
\usepackage{bussproofs}
\usepackage{amsmath}
\usepackage{mathtools}
\usepackage{amsfonts}
\usepackage{wasysym}
\usepackage{url}
\usepackage{float} 
\usepackage{subcaption}
\usepackage{pgfplots}
\pgfplotsset{compat=newest}
\usepgfplotslibrary{fillbetween}
\usepackage[shortlabels]{enumitem}
\usepackage{esint}
\usepackage{multicol}
\usepackage{todonotes}

\newtheorem{definition}{Definição}
\newtheorem{lema}{Lema}
\newtheorem{teorema}{Teorema}
\newtheorem{corolario}{Corolário}
\newtheorem*{obs}{Observação}

\setlength{\topmargin}{-1.0in}
\setlength{\oddsidemargin}{0in}
\setlength{\evensidemargin}{0in}
\setlength{\textheight}{10.5in}
\setlength{\textwidth}{6.5in}
\setlength{\baselineskip}{12mm}

\newcommand{\dx}{\ \mathrm{d} x }
\newcommand{\dy}{\ \mathrm{d} y }
\newcommand{\dz}{\ \mathrm{d} z }
\newcommand{\du}{\ \mathrm{d} u }
\newcommand{\dv}{\ \mathrm{d} v }
\newcommand{\dr}{\ \mathrm{d} r }
\newcommand{\dt}{\ \mathrm{d} t }
\newcommand{\dteta}{\ \mathrm{d} \theta }
\newcommand{\dro}{\ \mathrm{d} \rho }
\newcommand{\dfi}{\ \mathrm{d} \phi }
\newcommand{\ds}{\ \mathrm{d} s }
\newcommand{\dS}{\ \mathrm{d} S }
\newcommand{\dq}{\ \mathrm{d} q }
\newcommand{\dif}{\mathrm{d}}
\newcommand{\res}{\mathrm{res}}

\DeclareMathOperator{\rot}{rot}
\DeclareMathOperator{\diverg}{div}
\DeclareMathOperator{\tr}{tr}
\DeclareMathOperator{\rank}{rank}
\DeclareMathOperator{\Sym}{Sym}

\newcommand{\C}{\mathbb{C}}
\newcommand{\R}{\mathbb{R}}
\newcommand{\Q}{\mathbb{Q}}
\newcommand{\Z}{\mathbb{Z}}
\newcommand{\N}{\mathbb{N}}
\renewcommand{\P}{\mathbb{P}}
\newcommand{\F}{\mathcal{F}}
\renewcommand{\Re}{\mathrm{Re}}
\renewcommand{\Im}{\mathrm{Im}}

\graphicspath{{img/}}

\renewcommand{\sectionmark}[1]{ \markright{ \thesection.\ #1}}

\title{\textbf{Análise 2}\\ Exercícios Cap. 5}
\author{Caio Tomás de Paula}
\date{07/03/2022}
\begin{document}

\maketitle

\paragraph{Exercício 1.}\todo{terminar}
    %
    \begin{proof}
        Temos que
        %
        \begin{equation*}
            f'(x) = a + 2x\sin(1/x) - \cos(1/x), \forall x\in\R - \{0\}.
        \end{equation*}
        %
    \end{proof}
    %
\paragraph{Exercício 2.}\todo{terminar}
    %
    \begin{proof}
        Da convexidade de $U$, temos que $[a, a+h]\subset U$. Assim,
        podemos aplicar a desigualdade do valor médio e usar a limitação
        da derivada para obter
        %
        \begin{align*}
            |f(a+h) - f(a)| &\leq |h|\cdot\sup_{0<t<1} |f'(a+th)| \\
                            &\leq |h|M.
        \end{align*}
        %
        Daí, dado $\varepsilon >0$ podemos tomar $h$ tal que $|h| < \varepsilon/M$
        e, assim,
        %
        \begin{equation*}
            |f(a+h) - f(a)| \leq |h|M < \varepsilon,
        \end{equation*}
        %
        para todo $a\in U$. Logo, $f$ é uniformemente contínua em $U$.
    \end{proof}
    %
\paragraph{Exercício 3.}
    %
    \begin{proof}
        Segue da Proposição 5.1, da continuidade de $f'$ e da compacidade de $K$:
        pela proposição, temos que para cada $x\in K$ existe $c_x$ tal que
        %
        \begin{equation*}
            |f(x+h) - f(x)| \geq c_x|h|,
        \end{equation*}
        %
        para todo $|h|<\delta$. Daí, segue que
        %
        \begin{equation*}
            |f'(x)| \geq c_x \geq c,
        \end{equation*}
        %
        sendo
        %
        \begin{equation*}
            c = \min_{x\in K} c_x,
        \end{equation*}
        %
        que existe porque $f'$ é contínua e $K$ é compacto. Logo, para todo $x\in K$
        e $|h| < \delta$, vale
        %
        \begin{equation*}
            |f(x+h) - f(x)| \geq c|h|,
        \end{equation*}
        %
        como queríamos mostrar.
    \end{proof}
    %
\paragraph{Exercício 4.}\todo{terminar}
    %
    \begin{enumerate}[a)]
        \item Temos
        %
        \begin{equation*}
            d_U(a,b) = \inf\left\{ \sum_{i=1}^{k-1} |x_{i+1} - x_i| \right\},
        \end{equation*}
        %
        sendo $x_1, \dots, x_k\in\R^m$ vértices de uma poligonal que liga
        $a$ e $b$. Como os elementos do conjunto sobre o qual tomamos o ínfimo
        são todos maiores ou iguais a zero, segue que $d_U(a,b) \geq 0$ para
        todo par $(a,b)\in\R^m\times\R^m$.
        
        Ademais, se $a=b$ é evidente que o ínfimo é $0$, pois esse é o menor
        valor no conjunto. Reciprocamente, se
        $d_U(a,b) = 0$, então existe uma poligonal de comprimento nulo ligando
        $a$ e $b$, ou seja, $a=b$.
        
        Por fim, $d_U$ obedece a desigualdade triangular pois...
        
        No caso em que $U$ é convexo, sempre existe um segmento de reta inteiramente
        contido em $U$ que liga $a$ e $b$; pela desigualdade triangular,
        o comprimento desse segmento, $|a-b|$, é maior ou igual ao comprimento de
        qualquer poligonal ligando $a$ e $b$. Portanto, nesse caso temos
        $d_U(a,b) = |a-b|$.
        
        Por fim, para $U$ um aberto conexo qualquer, temos do exposto acima que
        %
        \begin{equation*}
            |a-b| \leq d_U(a,b),
        \end{equation*}
        %
        para quaisquer $a,b\in\R^m$. Agora, para cotar $d_U(a,b)$ por cima...
        
        \item Pela Desigualdade do Valor Médio, temos que se existe $M > 0$
        tal que $|f'(x)| \leq M$
        para todo $x\in U$ então
        %
        \begin{equation*}
            |f(x) - f(y)| \leq M|x-y| \leq M d_U(x,y),
        \end{equation*}
        %
        onde a última desigualdade segue do item anterior.
        
        \item Pela Desigualdade do Valor Médio e do fato que a derivada de $f$
        é limitada por $M$ em $V$, temos
        %
        \begin{equation*}
            d_V(f(x), f(y)) \leq |x-y| \sup_{0<t<1} \{ d_V(f'(y + t(x-y)), 0) \}
                            \leq M |x-y|,
        \end{equation*}
        %
        para quaisquer $x,y\in U$.
        
        \item
    \end{enumerate}
    %
\paragraph{Exercício 5.}\todo{terminar}
    %
    \begin{proof}
        Primeiro, supondo $E = \R^m$ com a norma do máximo e $F = \R^n$ com
        a norma do máximo, temos que
        %
        \begin{align*}
            |T| \geq |T\cdot e_j| = \max_i |t_{ij}|, \forall j = 1, \dots, m.
        \end{align*}
        %
        Logo,
        %
        \begin{align*}
            |T| \geq \max_{i,j} |t_{ij}|, \forall j = 1, \dots, m,
        \end{align*}
        %
        com $i = 1, \dots, n$. Por outro lado, considerando o vetor $v = (1,\dots,1)$
        em $\R^m$ (que tem norma $1$), temos que
        %
        \begin{align*}
            |T \cdot v|
            &= \left|\begin{bmatrix}
                \displaystyle{\sum_{j=1}^m t_{1j}} \\
                \vdots \\
                \displaystyle{\sum_{j=1}^m t_{nj}} \\
            \end{bmatrix}\right| \\
            &= \max_{1\leq i\leq n} \left| \displaystyle{\sum_{j=1}^m t_{ij}} \right| \\
            &\leq \max_i \displaystyle{\sum_{j=1}^m |t_{ij}|} \\
            &\leq \max_i \{m\cdot\max_j |t_{ij}|\} \\
            &= m\max_{i,j} |t_{ij}|.
        \end{align*}
        %
        Logo, como para qualquer outro vetor $u$ de norma $1$ em $\R^m$ teremos
        $|T\cdot u| \leq |T\cdot v|$, segue que
        %
        \begin{equation*}
            |T| \leq m \max_{i,j}|t_{ij}|.
        \end{equation*}
        %
        Agora, se tomarmos $E = \R^m$ com a norma da soma e $F = \R^n$ com
        a norma do máximo, teremos como antes que
        %
        \begin{align*}
            |T| \geq |T\cdot e_j| = \max_i |t_{ij}|, \forall j = 1, \dots, m.
        \end{align*}
        %
        mas agora, tomando $v = (1/m, \dots, 1/m)$, que é unitário em $\R^m$ 
        com a norma da soma, teremos
        %
        \begin{align*}
            |T \cdot v|
            &= \left|\begin{bmatrix}
                \displaystyle{\frac{1}{m}\sum_{j=1}^m t_{1j}} \\
                \vdots \\
                \displaystyle{\frac{1}{m}\sum_{j=1}^m t_{nj}} \\
            \end{bmatrix}\right| \\
            &\leq \frac{1}{m}\cdot\max_i \displaystyle{\sum_{j=1}^m |t_{ij}|} \\
            &\leq \frac{1}{m}\cdot\max_i \{m\cdot\max_j |t_{ij}|\} \\
            &= \max_{i,j} |t_{ij}|.
        \end{align*}
        %
        Logo, argumentando como acima, segue que
        %
        \begin{equation*}
            |T| \leq \max_{i,j} |t_{ij}|
        \end{equation*}
        %
        e, portanto,
        %
        \begin{equation*}
            T = \max_{i,j} |t_{ij}|.
        \end{equation*}
        %
    \end{proof}
    %
\paragraph{Exercício 6.}
    %
    \begin{proof}
        Note que em qualquer ponto $(x_0, t_0)\in\R^2$ com $t_0\neq 0$ teremos
        $F$ contínua, pois
        %
        \begin{align*}
            \lim_{(x,t)\to (x_0,t_0)} F(x,t)
            = \lim_{(x,t)\to (x_0,t_0)} \frac{f(x+t) - f(x)}{t}
            = \frac{f(x_0+t_0) - f(x_0)}{t_0}
            = F(x_0, t_0).
        \end{align*}
        %
        Em particular, $F$ é contínua em $x$. Daí, temos
        %
        \begin{align*}
            \lim_{(x,t) \to (x, 0)} F(x,t)
            = \lim_{t\to 0} \frac{f(x+t) - f(x)}{t}
            = f'(x)
            = F(x,0),
        \end{align*}
        %
        de modo que $F$ é contínua em $\R^2$.
    \end{proof}
    %
\paragraph{Exercício 7.}\todo{terminar}
    %
    A princípio, o maior $U$ que podemos tomar é
    %
    \begin{equation*}
        U = \mathcal{Q}_1 \cup \mathcal{Q}_3,
    \end{equation*}
    %
    ou seja, a união do primeiro e terceiro quadrantes (sem incluir os eixos).
    %
\paragraph{Exercício 8.}
    %
    a
    %
\paragraph{Exercício 9.}
    %
    a
    %
\paragraph{Exercício 10.}
    %
    a
    %
\paragraph{Exercício 11.}\todo{terminar}
    %
    \begin{proof}
        Dado $\varepsilon > 0$, defina
        %
        \begin{equation*}
            X = \{ t\in [0,1] : |f(t) - f(0)| \leq g(t) - g(0) + 
                                              \varepsilon t + \varepsilon \}.
        \end{equation*}
        %
        Note que $0\in X$, logo $X\neq\varnothing$. Ademais, dada uma sequência
        $(a_n)_{n\in\N}$ de pontos de $X$ que converge para $a$, temos que
        %
        \begin{equation*}
            |f(a_n) - f(0)| \leq g(a_n) - g(0) + \varepsilon a_n + \varepsilon, 
            \forall n\in\N.
        \end{equation*}
        %
        Tomando o limite e usando a diferenciabilidade de $f$ e de $g$, temos
        %
        \begin{equation*}
            |f(a) - f(0)| \leq g(a) - g(0) + \varepsilon a + \varepsilon,
        \end{equation*}
        %
        ou seja, $a\in X$ e, portanto, $X$ é fechado. Por fim, se
        $0 < a < 1$ e $a\in X$ vamos mostrar que existe $\delta > 0$
        tal que $a + \delta \in X$, ou seja, que $X$ é aberto.
        % terminar a demonstração
    \end{proof}
    %
\end{document}