\documentclass[12pt,a4paper]{article}
\usepackage{amsmath,amssymb,amsthm}
\usepackage{makeidx,graphics}
\usepackage[dvips]{graphicx}
\usepackage[portuguese]{babel}
\usepackage[utf8]{inputenc}
\usepackage{ae}
\usepackage{indentfirst}
\usepackage{amsbsy}
\usepackage{fancyhdr}
\usepackage{pstricks}
\usepackage[all]{xy}
\usepackage{wrapfig}
\usepackage[pdfstartview=FitH,backref,colorlinks,bookmarksnumbered,bookmarksopen,linktocpage,urlcolor=blue,linkcolor=cyan]{hyperref}
\usepackage{bussproofs}
\usepackage{amsmath}
\usepackage{mathtools}
\usepackage{amsfonts}
\usepackage{wasysym}
\usepackage{url}
\usepackage{float} 
\usepackage{subcaption}
\usepackage{pgfplots}
\pgfplotsset{compat=newest}
\usepgfplotslibrary{fillbetween}
\usepackage[shortlabels]{enumitem}
\usepackage{esint}
\usepackage{multicol}
\usepackage{todonotes}

\newtheorem{definition}{Definição}
\newtheorem{lema}{Lema}
\newtheorem{teorema}{Teorema}
\newtheorem{corolario}{Corolário}
\newtheorem*{obs}{Observação}

\setlength{\topmargin}{-1.0in}
\setlength{\oddsidemargin}{0in}
\setlength{\evensidemargin}{0in}
\setlength{\textheight}{10.5in}
\setlength{\textwidth}{6.5in}
\setlength{\baselineskip}{12mm}

\newcommand{\dx}{\ \mathrm{d} x }
\newcommand{\dy}{\ \mathrm{d} y }
\newcommand{\dz}{\ \mathrm{d} z }
\newcommand{\du}{\ \mathrm{d} u }
\newcommand{\dv}{\ \mathrm{d} v }
\newcommand{\dr}{\ \mathrm{d} r }
\newcommand{\dt}{\ \mathrm{d} t }
\newcommand{\dteta}{\ \mathrm{d} \theta }
\newcommand{\dro}{\ \mathrm{d} \rho }
\newcommand{\dfi}{\ \mathrm{d} \phi }
\newcommand{\ds}{\ \mathrm{d} s }
\newcommand{\dS}{\ \mathrm{d} S }
\newcommand{\dq}{\ \mathrm{d} q }
\newcommand{\dif}{\mathrm{d}}
\newcommand{\res}{\mathrm{res}}

\DeclareMathOperator{\rot}{rot}
\DeclareMathOperator{\diverg}{div}
\DeclareMathOperator{\tr}{tr}
\DeclareMathOperator{\rank}{rank}
\DeclareMathOperator{\Sym}{Sym}

\newcommand{\C}{\mathbb{C}}
\newcommand{\R}{\mathbb{R}}
\newcommand{\Q}{\mathbb{Q}}
\newcommand{\Z}{\mathbb{Z}}
\newcommand{\N}{\mathbb{N}}
\renewcommand{\P}{\mathbb{P}}
\newcommand{\F}{\mathcal{F}}
\renewcommand{\Re}{\mathrm{Re}}
\renewcommand{\Im}{\mathrm{Im}}

\graphicspath{{img/}}

\renewcommand{\sectionmark}[1]{ \markright{ \thesection.\ #1}}

\title{\textbf{Análise 2}\\ Exercícios Cap.8}
\author{Caio Tomás de Paula}
\date{28/03/2022}
\begin{document}

\maketitle

\paragraph{Exercício 1.}\todo{terminar}
%

% não sei provar a primeira parte ainda...

Tendo mostrado que
%
\begin{equation*}
    \left\langle
    \nu'(x) \cdot h, \varphi'(x) \cdot k
    \right\rangle
    =
    - \left\langle
    \nu(x), \varphi''(x) \cdot (h, k)
    \right\rangle,
\end{equation*}
%
segue que
%
\begin{equation*}
    \left\langle
    \nu'(x) \cdot k, \varphi'(x) \cdot h
    \right\rangle
    =
    - \left\langle
    \nu(x), \varphi''(x) \cdot (k, h)
    \right\rangle
\end{equation*}
%
e, da simetria de $\varphi''$, segue que
%
\begin{equation*}
    \left\langle
    \nu'(x) \cdot h, \varphi'(x) \cdot k
    \right\rangle
    =
    \left\langle
    \nu'(x) \cdot k, \varphi'(x) \cdot h
    \right\rangle.
\end{equation*}
%
Portanto, a aplicação linear $A: E_x \to E_x$ definida por $A = \nu'(x) \cdot [\varphi'(x)]^{-1}$
é auto-adjunta.
%
\paragraph{Exercício 2.}
%
Heuristicamente,
%
\begin{align*}
    D^2f(a) \cdot \vec{v} \cdot \vec{w}
    &\approx 
    Df(a+\vec{v})\vec{w} - Df(a)\vec{w} \\
    &\approx
    f(a + \vec{v} + \vec{w}) - f(a + \vec{v}) - f(a + \vec{w}) + f(a) \\
    &\approx 
    f(a + \vec{v} + \vec{w}) - f(a + \vec{w}) - f(a + \vec{v}) + f(a) \\
    &\approx
    Df(a + \vec{w})\vec{v} - Df(a)\vec{v} \\
    &\approx
    D^2f(a) \cdot \vec{w} \cdot \vec{v}.
\end{align*}
%
Mais precisamente,
%
\begin{align*}
    D^2f(a) \cdot \vec{v} \cdot \vec{w}
    &=
    \lim_{t_1 \to 0} \frac{Df(a + \vec{v} + t_1\vec{w}) - Df(a + t_1\vec{w})}{t} \\
    &=
    \lim_{t_1 \to 0} \frac{1}{t_1}
    \left[ 
    \lim_{t_2 \to 0} \frac{f(a + t_2\vec{v} + t_1\vec{w}) - f(a + t_2\vec{v}) 
    - f(a + t_1\vec{w}) + f(a)}{t_2}
    \right] \\
    &=
    \lim_{t \to 0} 
    \frac{f(a + t(\vec{v} + \vec{w})) - f(a + t\vec{v}) - f(a + t\vec{w}) + f(a)}{t^2}.
\end{align*}
%
Daí, segue que
%
\begin{align*}
    D^2f(a) \cdot \vec{w} \cdot \vec{v}
    &=
    \lim_{t \to 0} 
    \frac{f(a + t(\vec{w} + \vec{v})) - f(a + t\vec{w}) - f(a + t\vec{v}) + f(a)}{t^2} \\
    &=
    \lim_{t \to 0} 
    \frac{f(a + t(\vec{v} + \vec{w})) - f(a + t\vec{v}) - f(a + t\vec{w}) + f(a)}{t^2} \\
    &=
    D^2f(a) \cdot \vec{v} \cdot \vec{w},
\end{align*}
%
como queríamos mostrar.
\end{document}