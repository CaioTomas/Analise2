\documentclass[12pt,a4paper]{article}
\usepackage{amsmath,amssymb,amsthm}
\usepackage{makeidx,graphics}
\usepackage[dvips]{graphicx}
\usepackage[portuguese]{babel}
\usepackage[utf8]{inputenc}
\usepackage{ae}
\usepackage{indentfirst}
\usepackage{amsbsy}
\usepackage{fancyhdr}
\usepackage{pstricks}
\usepackage[all]{xy}
\usepackage{wrapfig}
\usepackage[pdfstartview=FitH,backref,colorlinks,bookmarksnumbered,bookmarksopen,linktocpage,urlcolor=blue,linkcolor=cyan]{hyperref}
\usepackage{bussproofs}
\usepackage{amsmath}
\usepackage{mathtools}
\usepackage{amsfonts}
\usepackage{wasysym}
\usepackage{url}
\usepackage{float} 
\usepackage{subcaption}
\usepackage{pgfplots}
\pgfplotsset{compat=newest}
\usepgfplotslibrary{fillbetween}
\usepackage[shortlabels]{enumitem}
\usepackage{esint}
\usepackage{multicol}
\usepackage{todonotes}

\newtheorem{definition}{Definição}
\newtheorem{lema}{Lema}
\newtheorem{teorema}{Teorema}
\newtheorem{corolario}{Corolário}
\newtheorem*{obs}{Observação}

\setlength{\topmargin}{-1.0in}
\setlength{\oddsidemargin}{0in}
\setlength{\evensidemargin}{0in}
\setlength{\textheight}{10.5in}
\setlength{\textwidth}{6.5in}
\setlength{\baselineskip}{12mm}

\newcommand{\dx}{\ \mathrm{d} x }
\newcommand{\dy}{\ \mathrm{d} y }
\newcommand{\dz}{\ \mathrm{d} z }
\newcommand{\du}{\ \mathrm{d} u }
\newcommand{\dv}{\ \mathrm{d} v }
\newcommand{\dr}{\ \mathrm{d} r }
\newcommand{\dt}{\ \mathrm{d} t }
\newcommand{\dteta}{\ \mathrm{d} \theta }
\newcommand{\dro}{\ \mathrm{d} \rho }
\newcommand{\dfi}{\ \mathrm{d} \phi }
\newcommand{\ds}{\ \mathrm{d} s }
\newcommand{\dS}{\ \mathrm{d} S }
\newcommand{\dq}{\ \mathrm{d} q }
\newcommand{\dif}{\mathrm{d}}
\newcommand{\res}{\mathrm{res}}

\DeclareMathOperator{\rot}{rot}
\DeclareMathOperator{\diverg}{div}

\graphicspath{{img/}}

\renewcommand{\sectionmark}[1]{ \markright{ \thesection.\ #1}}

\title{\textbf{Análise 2}\\ Exercícios Cap. 2}
\author{Caio Tomás de Paula}
\date{23/02/2022}
\begin{document}

\maketitle

\paragraph{Exercício 1.}
%
    \begin{enumerate}[a)]
        \item
        %
        \begin{proof}
            Temos que
            %
            \begin{align*}
                f(x+\Delta x, y+\Delta y, z+\Delta z) &= f(x,y,z) \\
                                                      &+(2x\Delta x - 2y\Delta y, x\Delta y + y\Delta x,
                                                        x\Delta z + z\Delta x, y\Delta z + z\Delta y) \\
                                                      &+ \rho(\Delta x, \Delta y, \Delta z),
            \end{align*}
            %
            sendo 
            %
            \begin{equation*}
                \rho(\Delta x, \Delta y, \Delta z) = (\Delta x^2 - \Delta y^2, \Delta x\Delta y, 
                                                    \Delta x\Delta z, \Delta y\Delta z),
            \end{equation*}
            %
            que é um infinitésimo (basta estimar a norma como fizemos num exercício parecido do Capítulo 1).
            Logo, $f$ é diferenciável em $\mathbb{R}^3$. A sua matriz jacobiana é
            %
            \begin{equation*}
                [Df(x,y,z)] = 
                \begin{bmatrix}
                    2x & -2y & 0 \\
                    y & x & 0 \\
                    z & 0 & x \\
                    0 & z & y
                \end{bmatrix}.
            \end{equation*}
            %
        \end{proof}
        %
        \item Note que, para $x, y\neq 0$, temos
        %
        \begin{align*}
            [Df(x,y,z)]\begin{bmatrix} a \\ b \\ c \end{bmatrix} 
            = [Df(x,y,z)]\begin{bmatrix} a' \\ b' \\ c' \end{bmatrix}
        \end{align*}
        %
        se, e somente se,
        %
        \begin{align*}
            \begin{cases}
                x(a-a') + y(b'-b) &= 0 \\
                y(a-a') + x(b-b') &= 0 \\
                z(a-a') + x(c-c') &= 0 \\
                z(b-b') + y(c-c') &= 0
            \end{cases}
            \iff 
            \begin{cases}
                a &= a' \\
                b &= b' \\
                c &= c'
            \end{cases},
        \end{align*}
        %
        ou seja, $f'(x,y,z)$ é injetiva. Note que se $x=0=y$, então temos apenas as condições
        %
        \begin{align*}
            \begin{cases}
                z(a-a') &= 0 \\
                z(b-b') &= 0
            \end{cases},
        \end{align*}
        %
        donde tiramos que $a = a'$ e $b = b'$, mas $c$ não precisa ser igual a $c'$.
        \item Temos
        %
        \begin{align*}
            f'(0,0,z)\begin{bmatrix} a \\ b \\ c \end{bmatrix} = \begin{bmatrix} 0 \\ 0 \\ az \\ bz \end{bmatrix},
        \end{align*}
        %
        de modo que a imagem de $f'(0,0,z)$ é o conjunto dos vetores da forma 
        $\displaystyle{ \begin{bmatrix} 0 \\ 0 \\ t \\ t \end{bmatrix}, t\in\mathbb{R} }$.
    \end{enumerate}
%
\paragraph{Exercício 2.}
    Temos que
    %
    \begin{align*}
        [f'(x,y)] = 
        \begin{bmatrix}
            e^x\cos y & -e^x\sin y \\
            e^x\sin y & e^x\cos y \\
        \end{bmatrix} 
        \implies
        [T] = [f'(3, \pi/6)] = 
        e^3A_{\pi/6}
    \end{align*}
    %
    sendo $A_{\pi/6}$ a matriz de rotação por $\pi/6$ radianos em torno da origem. Daí, temos que $[T^{100}]$
    é a matriz de rotação por $100\pi/6 \equiv 4\pi/6 = 2\pi/3$ radianos em torno da origem e $[T^{101}]$ é
    a matriz de rotação por $5\pi/6$ radianos em torno da origem. Portanto,
    %
    \begin{align*}
        T^{100}\cdot h = (\cos(2\pi/3), \sin(2\pi/3))
    \end{align*}
    %
%
\paragraph{Exercício 3.}
    a
%
\paragraph{Exercício 4.}
    a
%
\paragraph{Exercício 5.}
    a
%
\paragraph{Exercício 6.}
    a
%
\paragraph{Exercício 7.}
    a
%
\paragraph{Exercício 8.}
%
    \begin{enumerate}[a)]
        \item 
        %
        \item
        %
        \item 
    \end{enumerate}
%
\paragraph{Exercício 9.}
    a
%
\paragraph{Exercício 10.}
    a
%
\end{document}