\documentclass[12pt,a4paper]{article}
\usepackage{amsmath,amssymb,amsthm}
\usepackage{makeidx,graphics}
\usepackage[dvips]{graphicx}
\usepackage[portuguese]{babel}
\usepackage[utf8]{inputenc}
\usepackage{ae}
\usepackage{indentfirst}
\usepackage{amsbsy}
\usepackage{fancyhdr}
\usepackage{pstricks}
\usepackage[all]{xy}
\usepackage{wrapfig}
\usepackage[pdfstartview=FitH,backref,colorlinks,bookmarksnumbered,bookmarksopen,linktocpage,urlcolor=blue,linkcolor=cyan]{hyperref}
\usepackage{bussproofs}
\usepackage{amsmath}
\usepackage{mathtools}
\usepackage{amsfonts}
\usepackage{wasysym}
\usepackage{url}
\usepackage{float} 
\usepackage{subcaption}
\usepackage{pgfplots}
\pgfplotsset{compat=newest}
\usepgfplotslibrary{fillbetween}
\usepackage[shortlabels]{enumitem}
\usepackage{esint}
\usepackage{multicol}
\usepackage{todonotes}

\newtheorem{definition}{Definição}
\newtheorem{lema}{Lema}
\newtheorem{teorema}{Teorema}
\newtheorem{corolario}{Corolário}
\newtheorem*{obs}{Observação}

\setlength{\topmargin}{-1.0in}
\setlength{\oddsidemargin}{0in}
\setlength{\evensidemargin}{0in}
\setlength{\textheight}{10.5in}
\setlength{\textwidth}{6.5in}
\setlength{\baselineskip}{12mm}

\newcommand{\dx}{\ \mathrm{d} x }
\newcommand{\dy}{\ \mathrm{d} y }
\newcommand{\dz}{\ \mathrm{d} z }
\newcommand{\du}{\ \mathrm{d} u }
\newcommand{\dv}{\ \mathrm{d} v }
\newcommand{\dr}{\ \mathrm{d} r }
\newcommand{\dt}{\ \mathrm{d} t }
\newcommand{\dteta}{\ \mathrm{d} \theta }
\newcommand{\dro}{\ \mathrm{d} \rho }
\newcommand{\dfi}{\ \mathrm{d} \phi }
\newcommand{\ds}{\ \mathrm{d} s }
\newcommand{\dS}{\ \mathrm{d} S }
\newcommand{\dq}{\ \mathrm{d} q }
\newcommand{\dif}{\mathrm{d}}
\newcommand{\res}{\mathrm{res}}

\DeclareMathOperator{\rot}{rot}
\DeclareMathOperator{\diverg}{div}
\DeclareMathOperator{\tr}{tr}
\DeclareMathOperator{\rank}{rank}
\DeclareMathOperator{\Sym}{Sym}

\newcommand{\C}{\mathbb{C}}
\newcommand{\R}{\mathbb{R}}
\newcommand{\Q}{\mathbb{Q}}
\newcommand{\Z}{\mathbb{Z}}
\newcommand{\N}{\mathbb{N}}
\renewcommand{\P}{\mathbb{P}}
\newcommand{\F}{\mathcal{F}}
\renewcommand{\Re}{\mathrm{Re}}
\renewcommand{\Im}{\mathrm{Im}}

\graphicspath{{img/}}

\renewcommand{\sectionmark}[1]{ \markright{ \thesection.\ #1}}

\title{\textbf{Análise 2}\\ Exercícios Cap. 2}
\author{Caio Tomás de Paula}
\date{23/02/2022}
\begin{document}

\maketitle

\paragraph{Exercício 1.}
%
    \begin{enumerate}[a)]
        \item
        %
        \begin{proof}
            Temos que
            %
            \begin{align*}
                f(x+\Delta x, y+\Delta y, z+\Delta z) &= f(x,y,z) \\
                                                      &+(2x\Delta x - 2y\Delta y, x\Delta y + y\Delta x,
                                                        x\Delta z + z\Delta x, y\Delta z + z\Delta y) \\
                                                      &+ \rho(\Delta x, \Delta y, \Delta z),
            \end{align*}
            %
            sendo 
            %
            \begin{equation*}
                \rho(\Delta x, \Delta y, \Delta z) = (\Delta x^2 - \Delta y^2, \Delta x\Delta y, 
                                                    \Delta x\Delta z, \Delta y\Delta z),
            \end{equation*}
            %
            que é um infinitésimo (basta estimar a norma como fizemos num exercício parecido do Capítulo 1).
            Logo, $f$ é diferenciável em $\R^3$. A sua matriz jacobiana é
            %
            \begin{equation*}
                [Df(x,y,z)] = 
                \begin{bmatrix}
                    2x & -2y & 0 \\
                    y & x & 0 \\
                    z & 0 & x \\
                    0 & z & y
                \end{bmatrix}.
            \end{equation*}
            %
        \end{proof}
        %
        \item Note que, para $x, y\neq 0$, temos
        %
        \begin{align*}
            [Df(x,y,z)]\begin{bmatrix} a \\ b \\ c \end{bmatrix} 
            = [Df(x,y,z)]\begin{bmatrix} a' \\ b' \\ c' \end{bmatrix}
        \end{align*}
        %
        se, e somente se,
        %
        \begin{align*}
            \begin{cases}
                x(a-a') + y(b'-b) &= 0 \\
                y(a-a') + x(b-b') &= 0 \\
                z(a-a') + x(c-c') &= 0 \\
                z(b-b') + y(c-c') &= 0
            \end{cases}
            \iff 
            \begin{cases}
                a &= a' \\
                b &= b' \\
                c &= c'
            \end{cases},
        \end{align*}
        %
        ou seja, $f'(x,y,z)$ é injetiva. Note que se $x=0=y$, então temos apenas as condições
        %
        \begin{align*}
            \begin{cases}
                z(a-a') &= 0 \\
                z(b-b') &= 0
            \end{cases},
        \end{align*}
        %
        donde tiramos que $a = a'$ e $b = b'$, mas $c$ não precisa ser igual a $c'$.
        \item Temos
        %
        \begin{align*}
            f'(0,0,z)\begin{bmatrix} a \\ b \\ c \end{bmatrix} = \begin{bmatrix} 0 \\ 0 \\ az \\ bz \end{bmatrix},
        \end{align*}
        %
        de modo que a imagem de $f'(0,0,z)$ é o conjunto dos vetores da forma 
        $\displaystyle{ \begin{bmatrix} 0 \\ 0 \\ t \\ t \end{bmatrix}, t\in\R }$.
    \end{enumerate}
%
\paragraph{Exercício 2.}
    Temos que
    %
    \begin{align*}
        [f'(x,y)] = 
        \begin{bmatrix}
            e^x\cos y & -e^x\sin y \\
            e^x\sin y & e^x\cos y \\
        \end{bmatrix} 
        \implies
        [T] = [f'(3, \pi/6)] = 
        e^3A_{\pi/6}
    \end{align*}
    %
    sendo $A_{\pi/6}$ a matriz de rotação por $\pi/6$ radianos em torno da origem. Daí, temos que $[T^{100}]$
    é a matriz de rotação por $100\pi/6 \equiv 4\pi/6 = 2\pi/3$ radianos em torno da origem e $[T^{101}]$ é
    a matriz de rotação por $5\pi/6$ radianos em torno da origem. Portanto, o ângulo entre $T^{100}\cdot h$ 
    e $T^{101}\cdot k$ é
    %
    \begin{align*}
        \frac{\pi}{4} + \frac{5\pi}{6} - \frac{2\pi}{3} = \frac{13\pi}{12} - \frac{2\pi}{3} = \frac{5\pi}{12}.
    \end{align*}
    %
\paragraph{Exercício 3.}
    %
    \begin{proof}
        Usando que $f$ é diferenciável em $U$, temos que
        %
        \begin{align*}
            \varphi(x+h) &= (x+h, f(x+h)) \\
                         &= (x+h, f(x) + f'(x)\cdot h + \rho_f(h)) \\
                         &= \varphi(x) + (h, f'(x)\cdot h) + \rho_{\varphi}(h),
        \end{align*}
        %
        sendo $\rho_{\varphi}(h) = (0, \varphi_f(h))$ tal que 
        $\rho_{\varphi}(h)/\|h\| \xrightarrow{h\to \vec{0}} \vec{0}$ pois
        $\rho_f(h)/\|h\| \xrightarrow{h\to \vec{0}} 0$.
        Portanto, $\varphi$ é diferenciável e $\varphi'(x)\cdot h = (h, f'(x)\cdot h)$.
        Segue da expressão da derivada que $\varphi'$ é injetiva: de fato, dado $x\in U$,
        %
        \begin{equation*}
            \varphi'(x)\cdot h_1 = \varphi'(x)\cdot h_2 
            \implies (h_1, f'(x)\cdot h_1) = (h_2, f'(x)\cdot h_2)
            \implies h_1 = h_2.
        \end{equation*}
        %
        Agora para $F(x,y)$. Dado $(u,v)\in U\times\R^n$, temos
        %
        \begin{align*}
            F(x+u, y+v) &= f(x+u) - y - v \\
                        &= F(x,y) + f'(x)\cdot u - v + \rho_f(u).
        \end{align*}
        %
        Como $\rho_f(u)/\|h\| \xrightarrow{h\to\vec{0}} 0$, segue que $F'(x,y)\cdot (u,v) = f'(x)\cdot u - v$.
        Daí, o núcleo de $F'(x,y)$ são os $(u,v)$ tais que $v = f'(x)\cdot u$, ou seja, da forma
        $(u, f'(x)\cdot u)$, precisamente $\varphi'(U)$.
    \end{proof}
    %
\paragraph{Exercício 4.}
    %
    \begin{proof}
        Sabemos que
        %
        \begin{align*}
            [f'(x,y)] = 
            \begin{bmatrix}
                2x & 0 \\
                0 & 2y \\
                2(x+y) & 2(x+y)
            \end{bmatrix}.
        \end{align*}
        %
        Daí,
        %
        \begin{align*}
            [f'(x,y)]\cdot\vec{e}_1 &= 
            \begin{bmatrix}
                2x \\ 
                0 \\
                2(x+y)
            \end{bmatrix} \\
            [f'(x,y)]\cdot\vec{e}_2 = 
            \begin{bmatrix}
                0 \\ 
                2y \\
                2(x+y)
            \end{bmatrix}.
        \end{align*}
        %
        Portanto, se $a[f'(x,y)]\cdot\vec{e}_1 + b[f'(x,y)]\cdot\vec{e}_2 = \mathbf{0}$, então
        %
        \begin{align*}
            \begin{bmatrix}
                0 \\
                0 \\
                0
            \end{bmatrix}
            =
            \begin{bmatrix}
                2ax \\
                2by \\
                2(x+y)(a+b)
            \end{bmatrix}
        \end{align*}
        %
        que ocorre (considerando $x,y\neq 0$) se, e somente se $a=0=b$; logo, $f'(x,y)$ tem posto 2.
    \end{proof}
    %
\paragraph{Exercício 5.}
    %
    \begin{proof}
        Temos que
        %
        \begin{equation*}
            f'(x,y,z) =
            \begin{bmatrix}
                1 & 1 & 1 \\
                2x & 2y & 2z \\
                3x^2 & 3y^2 & 3z^2
            \end{bmatrix}.
        \end{equation*}
        %
        Daí,
        %
        \begin{align*}
            \det([f'(x,y,z)]) = 0 &\iff 6yz^2 + 6zx^2 + 6xy^2 - 6yx^2 - 6xz^2 - 6zy^2 = 0 \\
                                  &\iff y(z^2 - x^2) + z(x^2 - y^2) + x(y^2 - z^2) = 0 \\
                                  &\iff (x = y) \lor (x = z) \lor (y = z).
        \end{align*}
        %
        Portanto, $f'(x,y,z)$ é invertível salvo se duas das coordenadas forem iguais.
    \end{proof}
    %
\paragraph{Exercício 6.}
    %
    \begin{proof}
        Temos que
        %
        \begin{align*}
            [f'(x,y)] =
            \begin{bmatrix}
                e^x & e^y \\
                e^x & -e^y
            \end{bmatrix}.
        \end{align*}
        %
        Como $\det([f'(x,y)]) = -2e^{x+y} \neq 0, \forall (x,y)\in\R^2$, segue que $f'(x,y)$ é invertível
        para todo ponto do plano.
    \end{proof}
    %
    Pensando $f:\C\to\C$, temos que
    %
    \begin{align*}
        f(x+iy) = (e^x + e^y) + i(e^x - e^y) = u(x,y) + iv(x,y).   
    \end{align*}
    %
    Note que
    %
    \begin{align*}
        \frac{\partial u}{\partial x} = e^x &\neq -e^y = \frac{\partial v}{\partial y} \\
        \frac{\partial u}{\partial y} = e^y &\neq -e^x = -\frac{\partial v}{\partial x},
    \end{align*}
    %
    de modo que $f$, vista como função complexa, não é holomorfa, pois não satisfaz as condições de 
    Cauchy-Riemann.
    %
\paragraph{Exercício 7.}
    %
    \begin{proof}
        Dado $H\in E$, temos que
        %
        \begin{align*}
            f(X+H) &= (X+H)(X+H)^* \\
                   &= (X+H)(X^* + H^*) \\
                   &= f(X) + XH^* + HX^* + \rho(H),
        \end{align*}
        %
        sendo $\rho(H) = HH^*$. Ora, como $\rho(H)/H \xrightarrow{H\to 0} 0$, então $f'(X)\cdot H = XH^* + HX^*$.
        Note que
        %
        \begin{equation*}
            (f'(X)\cdot H)^* = (XH^* + HX^*)^* = (XH^*)^* + (HX^*)^* = HX^* + XH^* = f'(X)\cdot H,
        \end{equation*}
        %
        ou seja, $f'(X)\cdot H$ é simétrica para todo $H\in E$. Dito de outro modo, a imagem de $f'(X)$
        é o conjunto das matrizes simétricas de ordem $n$. Resta mostrar que se $X$ é ortogonal, então
        $f'(X)$ é sobrejetora.
        
        Por último, queremos mostrar que $f'(X)$ é sobrejetora para $X$ ortogonal. 
        
        O exposto acima nos diz que
        %
        \begin{equation*}
            \Im(f'(X)) \subset \Sym(n,\R),
        \end{equation*}
        %
        sendo $\Sym(n,\R)$ o conjunto das matrizes simétricas de ordem $n$. Portanto, para mostrar
        o que queremos, basta mostrar que a dimensão de $f'(X)$ para $X$ ortogonal é igual à 
        dimensão de $\Sym(n,\R)$.
        
        Em $X = I_n$ temos nossa vida facilitada, pois se $S\in\Sym(n,\R)$, então
        %
        \begin{equation*}
            f'(I)\cdot (S/2) = S^*/2 + S/2 = S.
        \end{equation*}
        %
        Para o caso de $X$ geral, defina
        %
        \begin{equation*}
            \varphi_X: A\mapsto AX,
        \end{equation*}
        %
        a multiplicação à direita por $X$. Temos $\varphi_X$ transformação linear, de modo que
        $\varphi'_X(\cdot) = \varphi_X$. Ademais, para $X$ ortogonal,
        %
        \begin{equation*}
            f(\varphi_X(A)) = AXX^*A^* = AA^* = f(A),
        \end{equation*}
        %
        ou seja, $f\circ\varphi_X = f$. Portanto, para $X$ ortogonal, temos pela regra da cadeia que
        %
        \begin{equation*}
            f'(I_n) = (f\circ\varphi_X)'(I_n) = f'(X)\circ\varphi_X'(I_n) = f'(X)\circ\varphi_X.
        \end{equation*}
        %
        Daí, como $\varphi_X$ é bijeção (pois $X$ é invertível), segue que a dimensão da imagem de $f'(X)$ 
        é igual à dimensão da imagem de $f'(I)$. Portanto, $\Im(f'(X)) = \Sym(n,\R)$.
    \end{proof}
    %
\paragraph{Exercício 8.}
%
    \begin{enumerate}[a)]
        \item 
        %
        \begin{proof}
            Analogamente ao caso de uma transformação linear, temos que $\exists M\in\R$ tal que
            %
            \begin{equation*}
                f(a_1, \dots, a_p) \leq M\|a_1\|\cdots\|a_p\|.
            \end{equation*}
            %
            Daí,
            %
            \begin{align*}
                f(a_1 + v_1, \dots, a_p + v_p) &= f(a_1, a_2,\dots, a_p) 
                                               + f(v_1,a_2, \dots, a_p) 
                                               + \cdots 
                                               + f(a_1, a_2, \dots, v_p) \\
                                               &+ \underbrace{f(v_1, v_2, a_3, \dots, a_p) 
                                               + \cdots 
                                               + f(a_1, a_2\dots, a_{n-2}, v_{n-1},v_n)}_{
                                               \text{infinitésimo}},
            \end{align*}
            %
            em que o fato do termo destacado ser um infinitésimo segue do fato de que cada uma das parcelas,
            dividida por $\|(v_1, \dots, v_p)\|_{\infty}$, é menor ou igual a
            %
            \begin{equation*}
                M\frac{\|a_3\|\cdots\|a_p\|\|v\|^2_{\infty}}{\|v\|_{\infty}} \xrightarrow{v\to 0} 0
            \end{equation*}
            %
            Logo, 
            %
            \begin{equation*}
                f'(a_1, \dots, a_p)\cdot (v_1, \dots, v_p) = f(v_1, a_2, \dots, a_p) 
                                                           + \cdots 
                                                           + f(a_1, a_2, \dots, v_p).
            \end{equation*}
            %
        \end{proof}
        %
        \item
        %
        \begin{proof}
            Ora, o determinante é uma função $n$-linear. Logo, 
            %
            \begin{align*}
                f'(X)\cdot H &= 
                \det\begin{bmatrix}
                        h_{11} & x_{12} & \cdots & x_{1n} \\
                        h_{21} & x_{22} & \cdots & x_{1n} \\
                        \vdots & \vdots & \ddots & \vdots \\
                        h_{n1} & x_{n2} & \cdots & x_{nn} \\
                    \end{bmatrix}
                    + \cdots + 
                \det\begin{bmatrix}
                        x_{11} & x_{12} & \cdots & h_{1n} \\
                        x_{21} & x_{22} & \cdots & h_{2n} \\
                        \vdots & \vdots & \ddots & \vdots \\
                        x_{n1} & x_{n2} & \cdots & h_{nn} \\
                    \end{bmatrix}.
            \end{align*}
            %
            Daí, segue que
            %
            \begin{align*}
                f'(I)\cdot H 
                &= 
                \det\begin{bmatrix}
                        h_{11} & 0 & \cdots & 0 \\
                        h_{21} & 1 & \cdots & 0 \\
                        \vdots & \vdots & \ddots & \vdots \\
                        h_{n1} & 0 & \cdots & 1 \\
                    \end{bmatrix}
                    + \cdots + 
                \det\begin{bmatrix}
                        1 & 0 & \cdots & h_{1n} \\
                        0 & 1 & \cdots & h_{2n} \\
                        \vdots & \vdots & \ddots & \vdots \\
                        0 & 0 & \cdots & h_{nn} \\
                    \end{bmatrix} \\
                &= h_{11} + \cdots + h_{nn} \\
                &= \tr(H).
            \end{align*}
            %
        \end{proof}
        %
        \item 
        %
        \begin{proof}
            Temos que $\det'(X) = 0$ se, e somente se, para todo $H\in E$, vale
            %
            \begin{equation*}
                \det\begin{bmatrix}
                        h_{11} & x_{12} & \cdots & x_{1n} \\
                        h_{21} & x_{22} & \cdots & x_{1n} \\
                        \vdots & \vdots & \ddots & \vdots \\
                        h_{n1} & x_{n2} & \cdots & x_{nn} \\
                    \end{bmatrix}
                    + \cdots + 
                \det\begin{bmatrix}
                        x_{11} & x_{12} & \cdots & h_{1n} \\
                        x_{21} & x_{22} & \cdots & h_{2n} \\
                        \vdots & \vdots & \ddots & \vdots \\
                        x_{n1} & x_{n2} & \cdots & h_{nn} \\
                    \end{bmatrix} = 0.
            \end{equation*}
            %
            Se $\rank(X)\leq n-2$, então pelo menos duas colunas de $X$ são combinações lineares
            das demais; logo, ainda que tiremos uma delas, a outra restará e, por isso, teremos
            $\det'(X)\cdot H = 0, \forall H\in E$ devido à forma dessa derivada.
            
            Por outro lado, se $\det'(X)\cdot H = 0, \forall H\in E$, então em todas as parcelas da soma
            acima temos pelo menos uma coluna que é combinação linear das demais. Como $H$ é arbitrário,
            segue que essas colunas pertencem a $X$. Como cada uma das parcelas contém uma tal coluna
            de $X$ que é combinação linear das demais colunas de $X$, segue que deve haver pelo menos
            duas dessas colunas em $X$ pois, se houvesse apenas uma, digamos a primeira, então a primeira
            parcela da soma acima não seria nula para todo $H$. Portanto, $\rank(X)\leq n-2$.
        \end{proof}
        %
    \end{enumerate}
%
\paragraph{Exercício 9.}
    %
    \begin{proof}
        Se $f$ tem um mínimo (máximo) relativo em $x\in U$, então existe uma vizinhança $V\subset U$ de 
        $x$ tal que $f(x) \leq f(x'), \forall x'\in V$. Como $f$ é diferenciável em $x$, temos que
        %
        \begin{equation*}
            f'(x) = \lim_{h\to 0} \frac{f(x + h) - f(x)}{h}.
        \end{equation*}
        %
        Para todo $h\in V$, temos que $f(x+h) - f(x) \geq 0$. Daí, segue que
        %
        \begin{align*}
            \lim_{h\to 0^-} \frac{f(x + h) - f(x)}{h} \leq 0 \leq \lim_{h\to 0^+} \frac{f(x + h) - f(x)}{h},
        \end{align*}
        %
        o que implica $f'(x) = 0$. Para um ponto de máximo relativo temos um raciocínio análogo, mas
        agora $f(x+h) - f(x) \leq 0$ e
        %
        \begin{align*}
            \lim_{h\to 0^+} \frac{f(x + h) - f(x)}{h} \leq 0 \leq \lim_{h\to 0^-} \frac{f(x + h) - f(x)}{h},
        \end{align*}
        %
        e segue novamente que $f'(x) = 0$.
    \end{proof}
    %
\paragraph{Exercício 10.}
    %
    \begin{proof}
        Esse resultado parece uma espécie de generalização do teorema de Rolle para o caso $m=1$.
        Intuitivamente, o fato da função ir para zero nos pontos de acumulação faz com que $f$ não
        possa crescer nem decrescer indefinidamente e isso implica que a derivada se anula em
        algum momento.
        
        Na verdade, o enunciado nos diz que $f$ pode ser estendida continuamente a 
        $\widetilde{f}:\overline{U}\to\R$ definindo $\widetilde{f}(u) = f(u)$ para $u\in U$
        e $f(a) = 0$ para $a\in\overline{U}\setminus U$. Como $U$ era aberto e limitado, temos
        $\overline{U}$ fechado e limitado de $\R^m$, ou seja, um compacto.
        Portanto, $\widetilde{f}$ assume máximo e mínimo em $\overline{U}$.
        
        Se ambos valores forem $0$, então $\widetilde{f}$ é constante e $f'(x) = 0, \forall x\in U$.
        Do contrário, existe $x_0\in U$ tal que $f(x_0)$ é um extremo (máximo ou mínimo). Daí,
        segue que as derivadas direcionais em cada direção $\vec{v}$ se anulam, pois
        $t=0$ é ponto extremo de $t\mapsto f(x_0 + t\vec{v})$ (que é uma função diferenciável de
        $\R$ em $\R$ e podemos usar o exercício anterior).
        
        Portanto, $\partial_{\vec{v}} f(x_0) = 0$ para todo $\vec{v}$, ou seja, $f'(x_0) = 0$.
    \end{proof}
    %
\end{document}