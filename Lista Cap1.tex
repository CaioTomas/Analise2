\documentclass[12pt,a4paper]{article}
\usepackage{amsmath,amssymb,amsthm}
\usepackage{makeidx,graphics}
\usepackage[dvips]{graphicx}
\usepackage[portuguese]{babel}
\usepackage[utf8]{inputenc}
\usepackage{ae}
\usepackage{indentfirst}
\usepackage{amsbsy}
\usepackage{fancyhdr}
\usepackage{pstricks}
\usepackage[all]{xy}
\usepackage{wrapfig}
\usepackage[pdfstartview=FitH,backref,colorlinks,bookmarksnumbered,bookmarksopen,linktocpage,urlcolor=blue,linkcolor=cyan]{hyperref}
\usepackage{bussproofs}
\usepackage{amsmath}
\usepackage{mathtools}
\usepackage{amsfonts}
\usepackage{wasysym}
\usepackage{url}
\usepackage{float} 
\usepackage{subcaption}
\usepackage{pgfplots}
\pgfplotsset{compat=newest}
\usepgfplotslibrary{fillbetween}
\usepackage[shortlabels]{enumitem}
\usepackage{esint}
\usepackage{multicol}
\usepackage{todonotes}

\newtheorem{definition}{Definição}
\newtheorem{lema}{Lema}
\newtheorem{teorema}{Teorema}
\newtheorem{corolario}{Corolário}
\newtheorem*{obs}{Observação}

\setlength{\topmargin}{-1.0in}
\setlength{\oddsidemargin}{0in}
\setlength{\evensidemargin}{0in}
\setlength{\textheight}{10.5in}
\setlength{\textwidth}{6.5in}
\setlength{\baselineskip}{12mm}

\newcommand{\dx}{\ \mathrm{d} x }
\newcommand{\dy}{\ \mathrm{d} y }
\newcommand{\dz}{\ \mathrm{d} z }
\newcommand{\du}{\ \mathrm{d} u }
\newcommand{\dv}{\ \mathrm{d} v }
\newcommand{\dr}{\ \mathrm{d} r }
\newcommand{\dt}{\ \mathrm{d} t }
\newcommand{\dteta}{\ \mathrm{d} \theta }
\newcommand{\dro}{\ \mathrm{d} \rho }
\newcommand{\dfi}{\ \mathrm{d} \phi }
\newcommand{\ds}{\ \mathrm{d} s }
\newcommand{\dS}{\ \mathrm{d} S }
\newcommand{\dq}{\ \mathrm{d} q }
\newcommand{\dif}{\mathrm{d}}
\newcommand{\res}{\mathrm{res}}

\DeclareMathOperator{\rot}{rot}
\DeclareMathOperator{\diverg}{div}
\DeclareMathOperator{\tr}{tr}
\DeclareMathOperator{\rank}{rank}

\newcommand{\C}{\mathbb{C}}
\newcommand{\R}{\mathbb{R}}
\newcommand{\Q}{\mathbb{Q}}
\newcommand{\Z}{\mathbb{Z}}
\newcommand{\N}{\mathbb{N}}
\renewcommand{\P}{\mathbb{P}}
\newcommand{\F}{\mathcal{F}}
\renewcommand{\Re}{\mathrm{Re}}
\renewcommand{\Im}{\mathrm{Im}}

\graphicspath{{img/}}

\renewcommand{\sectionmark}[1]{ \markright{ \thesection.\ #1}}

\title{\textbf{Análise 2}\\ Exercícios Cap. 1}
\author{Caio Tomás de Paula}
\date{18/02/2022}
\begin{document}

\maketitle

\paragraph{Exercício 1.}
%
    \begin{enumerate}[a)]
        \item Temos
        %
        \begin{align*}
            f'(z)\cdot h &= \lim_{t\to 0} \frac{f(4+t, -1+2t) - f(4,-1)}{t} \\
                         &= \lim_{t\to 0} \frac{((4+t)^2 -1+2t, 4+t+(-1+2t)^2) - (15,5)}{t} \\
                         &= \lim_{t\to 0} \frac{(t^2+10t, 4t^2-3t)}{t} \\
                         &= \lim_{t\to 0} \frac{(t+10,4t-3)}{t} \\
                         &= (10,-3).
        \end{align*}
        %
        \item Temos
        %
        \begin{align*}
            \varphi'(x)\cdot v &= \lim_{t\to 0} \frac{\varphi(x+tv) - \varphi(x)}{t} \\
                               &= \lim_{t\to 0} \frac{1}{t}\cdot\left[ (f(x) + f(tv))(g(x) + g(tv)) -
                               f(x)\cdot g(x) \right] \\
                               &= \lim_{t\to 0} [f(x)\cdot g(v) + g(x)\cdot f(v) + tf(v)\cdot g(v)] \\
                               &= f(x)\cdot g(v) + g(x)\cdot f(v),
        \end{align*}
        %
        usando o fato que $f$ e $g$ são funcionais lineares.
        \item Temos
        %
        \begin{align*}
            \xi'(x)\cdot h &= \lim_{t\to 0} \frac{\xi(x+th) - \xi(x)}{t} \\
                           &= \lim_{t\to 0} \frac{1}{t}\cdot\left[ \langle f(x+th), g(x+th) \rangle 
                           - \langle f(x), g(x) \rangle \right] \\
                           &= \lim_{t\to 0} \frac{1}{t}\left[ \langle f(x+th), g(x+th) \rangle
                           - \langle f(x), g(x+th) \rangle
                           + \langle f(x), g(x+th) \rangle 
                           - \langle f(x), g(x) \rangle \right] \\
                           &= \lim_{t\to 0} \left\langle \frac{f(x+th) - f(x)}{t}, g(x+th) \right\rangle
                           + \left\langle f(x), \frac{g(x+th) - g(x)}{t} \right\rangle \\
                           &= \langle f'(x)\cdot h, g(x) \rangle + \langle f(x), g'(x)\cdot h \rangle
        \end{align*}
        %
        novamente usando o fato de que $f$ e $g$ são funcionais lineares e propriedades do produto interno
        (inclusive que ele é contínuo).
    \end{enumerate}
%
\paragraph{Exercício 2.}
%
    \begin{enumerate}[a)]
        \item 
        %
        \begin{proof}
            Temos que
            %
            \begin{align*}
                f(x+\Delta x, y+\Delta y) &= (x^2 + 2x\Delta x + \Delta x^2 + y + \Delta y,
                                                x + \Delta x + y^2 + 2y\Delta y + \Delta y^2) \\
                                          &= f(x,y) + (2x\Delta x + \Delta y, \Delta x + 2y\Delta y)
                                          + (\Delta x^2, \Delta y^2) \\
                                          &= f(x,y)+f'(x)\cdot (\Delta x, \Delta y)+\rho(\Delta x, \Delta y),
            \end{align*}
            %
            sendo $f'(x) = \begin{bmatrix} 2x & 1 \\ 1 & 2y \end{bmatrix}$ a candidato a derivada. Ora, mas
            %
            \begin{equation*}
                \frac{\rho(\Delta x, \Delta y)}{\|(\Delta x, \Delta y)\|} 
                \xrightarrow{(\Delta x, \Delta y)\to\vec{0}} \vec{0}
            \end{equation*}
            %
            pois
            %
            \begin{equation*}
                \frac{\Delta x^2}{\|(\Delta x, \Delta y)\|} \leq \frac{\Delta x^2}{\Delta x}
                \xrightarrow{(\Delta x, \Delta y)\to\vec{0}} 0
            \end{equation*}
            %
            e analogamente para $\Delta y$.
        \end{proof}
        %
        \item
        %
        \begin{proof}
            Temos que
            %
            \begin{align*}
                \varphi(x+\vec{v}) &= f(x+\vec{v})\cdot g(x+\vec{v}) \\
                                   &= [f(x) + f(\vec{v})][g(x) + g(\vec{v})] \\
                                   &= \varphi(x) + f(x)\cdot g(\vec{v}) + g(x)\cdot f(\vec{v}) + f(\vec{v})\cdot g(\vec{v}).
            \end{align*}
            %
            Como $f,g$ são funcionais lineares, segue que
            %
            \begin{equation*}
                \frac{f(\vec{v})\cdot g(\vec{v})}{\|\vec{v}\|} \xrightarrow{\vec{v}\to \vec{0}} 0
            \end{equation*}
            %
            e, portanto, a derivada $\varphi'(x)\cdot \vec{v}$ existe e é 
            $f(x)\cdot g(\vec{v}) + g(x)\cdot f(\vec{v})$.
        \end{proof}
        %
        \item 
        %
        \begin{proof}
            Usando que $f,g$ são diferenciáveis e sendo $\rho_f, \rho_g$ os respectivos erros nas
            aproximações pelas derivadas, temos que 
            %
            \begin{align*}
                \xi(x+\vec{h}) &= \left\langle f(x+\vec{h}), g(x+\vec{h}) \right\rangle \\
                               &= \xi(x) + \left\langle f'(x)\cdot\vec{h}, g(x) \right\rangle 
                               + \left\langle f(x), g'(x)\cdot\vec{h} \right\rangle
                               + \rho_{\xi}(\vec{h}),
            \end{align*}
            %
            sendo
            %
            \begin{align*}
                \rho_{\xi}(\vec{h}) &= \left\langle f'(x)\cdot\vec{h}, g'(x)\cdot\vec{h} \right\rangle \\
                                    &+ \left\langle f'(x)\cdot\vec{h}, \rho_g(\vec{h}) \right\rangle \\
                                    &+ \left\langle \rho_f(\vec{h}), g'(x)\cdot\vec{h} \right\rangle \\
                                    &+ \left\langle f(x), \rho_g(\vec{h}) \right\rangle \\
                                    &+ \left\langle \rho_f(\vec{h}), g(x) \right\rangle \\
                                    &+ \left\langle \rho_f(\vec{h}), \rho_g(\vec{h}) \right\rangle.
            \end{align*}
            %
            Como $f$ e $g$ são diferenciáveis, segue que $\rho_{\xi}(\vec{h})$ é um infinitésimo com
            respeito a $\vec{h}$ e, portanto, $\xi'(x)\cdot\vec{h} = \left\langle f'(x)\cdot\vec{h}, 
            g(x) \right\rangle + \left\langle f(x), g'(x)\cdot\vec{h} \right\rangle$.
        \end{proof}
        %
    \end{enumerate}
%
\paragraph{Exercício 3.}
    %
    \begin{proof}
        Temos que
        %
        \begin{align*}
            \lim_{t\to 0} \frac{r(0+th) - r(0)}{t} &= \lim_{t\to 0} \frac{r(th)}{t} \\
                                                   &= \lim_{t\to 0} \frac{1}{t}[f(x_0 + th) - f(x_0) 
                                                   - f'(x_0)\cdot th] \\
                                                   &= f'(x_0)\cdot h - f'(x_0)\cdot h \\
                                                   &= 0,
        \end{align*}
        %
        onde o limite existe pois $f$ é diferenciável. Logo, $r$ é diferenciável na origem (e a derivada 
        vale $0$).
    \end{proof}
    %
\paragraph{Exercício 4.}\todo{fazer}
    a
%
\paragraph{Exercício 5.}
    Usando o método do Exercício 1, temos que o candidato a derivada é
    %
    \begin{align*}
        f'(X)\cdot\vec{h} &= \lim_{t\to 0}\frac{1}{t}\left[ (X+t\vec{h})^3 - X^3 \right] \\
                          &= \lim_{t\to 0}\frac{1}{t}\left[ tX^2\cdot\vec{h} + tX\cdot\vec{h}\cdot X
                          + t^2X\cdot\vec{h}^2 + t\vec{h}\cdot X^2 + t^2\vec{h}\cdot X\cdot\vec{h}
                          + t^2\vec{h}^2\cdot X + t^3\vec{h}^3 \right] \\
                          &= X^2\cdot\vec{h} + X\cdot\vec{h}\cdot X + \vec{h}\cdot X^2.
    \end{align*}    
    %
    Vamos, então, mostrar que esse é o caso.
    %
    \begin{proof}
        Temos
        %
        \begin{align*}
            f(X+\vec{h}) = (X+\vec{h})^3 &= X^3 + X^2\cdot\vec{h} + X\cdot\vec{h}\cdot X
                                         + X\cdot\vec{h}^2 + \vec{h}\cdot X^2 + \vec{h}\cdot X\cdot\vec{h}
                                         + \vec{h}^2\cdot X + \vec{h}^3 \\
                                         &= f(X) + f'(X)\cdot\vec{h} + \rho(\vec{h}), \, \forall\vec{h}
        \end{align*}
        %
        sendo $f'(X)\cdot\vec{h}$ o candidato a derivada e $\rho(\vec{h}) = X\cdot\vec{h}^2 + 
        \vec{h}\cdot X\cdot\vec{h} + \vec{h}^2\cdot X + \vec{h}^3$. Ora, mas 
        $\rho(\vec{h})/\|\vec{h}\| \xrightarrow{\vec{h}\to\vec{0}} \vec{0}$ (basta estimar superiormente
        cada uma das parcelas), de modo que $f$ é diferenciável em todos os pontos de $E$.
    \end{proof}
    %
%
\paragraph{Exercício 6.}
    Dados $E,F$ espaços vetoriais normados quaisquer, dizemos que $f:E\to F$ é diferenciável em $a\in E$ 
    se podemos escrever
    %
    \begin{equation*}
        f(a+\vec{v}) = f(a) + T\vec{v} + \rho(\vec{v}),
    \end{equation*}
    %
    sendo $T\in\mathcal{L}(E,F)$ com $\rho(\vec{v})/\|\vec{v}\| \xrightarrow{\vec{v}\to\vec{0}} \vec{0}$.
    Nesse caso, $T = f'(a)$. Vamos mostrar que $f$ ser contínua em $x$ é equivalente à continuidade de
    $T$ em $x$.
    %
    \begin{proof}
        Temos $f$ contínua em $x$ se, e somente se
        %
        \begin{equation*}
            \lim_{a\to x} f(a) = f(x).
        \end{equation*}
        %
        Usando que $f$ é diferenciável em $x$, segue que $f$ é contínua em $x$ se, e somente se,
        %
        \begin{align*}
            f(x+\vec{v}) 
            = f(x) + \lim_{a\to x} f'(a)\cdot\vec{v} + \rho(\vec{v}) 
            = f(x) + f'(x)\cdot\vec{v} + \rho(\vec{v}),
        \end{align*}
        %
        ou seja, se, e só se, $f'(a) \xrightarrow{a\to x} f'(x)$, i.e., $T$ é contínua em $x$.
    \end{proof}
    %
%
\paragraph{Exercício 7.}
    \begin{proof}
        Note que, sendo $h = x\cdot e^{iy}$, temos
        %
        \begin{equation*}
            \lim_{t\to 0} \frac{f(th)}{t} = \lim_{t\to 0} \frac{(tx/y)^2}{t} = \lim_{t\to 0} t(x/y)^2 = 0.
        \end{equation*}
        %
        Entretanto, $f$ não é contínua na origem. De fato, temos que $f'(x\cdot e^{iy})=[2x/y^2, -2x^2/y^3]$,
        que não é contínua na origem pois quando fixamos $x$ e fazemos $y$ tender a zero, o limite não existe.
        Logo, segue do Exercício 6 que $f$ não é contínua na origem.
    \end{proof}
%
\end{document}