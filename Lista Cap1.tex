\documentclass[12pt,a4paper]{article}
\usepackage{amsmath,amssymb,amsthm}
\usepackage{makeidx,graphics}
\usepackage[dvips]{graphicx}
\usepackage[portuguese]{babel}
\usepackage[utf8]{inputenc}
\usepackage{ae}
\usepackage{indentfirst}
\usepackage{amsbsy}
\usepackage{fancyhdr}
\usepackage{pstricks}
\usepackage[all]{xy}
\usepackage{wrapfig}
\usepackage[pdfstartview=FitH,backref,colorlinks,bookmarksnumbered,bookmarksopen,linktocpage,urlcolor=blue,linkcolor=cyan]{hyperref}
\usepackage{bussproofs}
\usepackage{amsmath}
\usepackage{mathtools}
\usepackage{amsfonts}
\usepackage{wasysym}
\usepackage{url}
\usepackage{float} 
\usepackage{subcaption}
\usepackage{pgfplots}
\pgfplotsset{compat=newest}
\usepgfplotslibrary{fillbetween}
\usepackage[shortlabels]{enumitem}
\usepackage{esint}
\usepackage{multicol}
\usepackage{todonotes}

\newtheorem{definition}{Definição}
\newtheorem{lema}{Lema}
\newtheorem{teorema}{Teorema}
\newtheorem{corolario}{Corolário}
\newtheorem*{obs}{Observação}

\setlength{\topmargin}{-1.0in}
\setlength{\oddsidemargin}{0in}
\setlength{\evensidemargin}{0in}
\setlength{\textheight}{10.5in}
\setlength{\textwidth}{6.5in}
\setlength{\baselineskip}{12mm}

\newcommand{\dx}{\ \mathrm{d} x }
\newcommand{\dy}{\ \mathrm{d} y }
\newcommand{\dz}{\ \mathrm{d} z }
\newcommand{\du}{\ \mathrm{d} u }
\newcommand{\dv}{\ \mathrm{d} v }
\newcommand{\dr}{\ \mathrm{d} r }
\newcommand{\dt}{\ \mathrm{d} t }
\newcommand{\dteta}{\ \mathrm{d} \theta }
\newcommand{\dro}{\ \mathrm{d} \rho }
\newcommand{\dfi}{\ \mathrm{d} \phi }
\newcommand{\ds}{\ \mathrm{d} s }
\newcommand{\dS}{\ \mathrm{d} S }
\newcommand{\dq}{\ \mathrm{d} q }
\newcommand{\dif}{\mathrm{d}}
\newcommand{\res}{\mathrm{res}}

\DeclareMathOperator{\rot}{rot}
\DeclareMathOperator{\diverg}{div}

\graphicspath{{img/}}

\renewcommand{\sectionmark}[1]{ \markright{ \thesection.\ #1}}

\title{\textbf{Análise 2}\\ Exercícios Cap. 1}
\author{Caio Tomás de Paula}
\date{18/02/2022}
\begin{document}

\maketitle

\paragraph{Exercício 1.}
%
    \begin{enumerate}[a)]
        \item Temos
        %
        \begin{align*}
            f'(z)\cdot h &= \lim_{t\to 0} \frac{f(4+t, -1+2t) - f(4,-1)}{t} \\
                         &= \lim_{t\to 0} \frac{((4+t)^2 -1+2t, 4+t+(-1+2t)^2) - (15,5)}{t} \\
                         &= \lim_{t\to 0} \frac{(t^2+10t, 4t^2-3t)}{t} \\
                         &= \lim_{t\to 0} \frac{(t+10,4t-3)}{t} \\
                         &= (10,-3).
        \end{align*}
        %
        \item Temos
        %
        \begin{align*}
            \varphi'(x)\cdot v &= \lim_{t\to 0} \frac{\varphi(x+tv) - \varphi(x)}{t} \\
                               &= \lim_{t\to 0} \frac{1}{t}\cdot\left[ (f(x) + f(tv))(g(x) + g(tv)) -
                               f(x)\cdot g(x) \right] \\
                               &= \lim_{t\to 0} [f(x)\cdot g(v) + g(x)\cdot f(v) + tf(v)\cdot g(v)] \\
                               &= f(x)\cdot g(v) + g(x)\cdot f(v),
        \end{align*}
        %
        usando o fato que $f$ e $g$ são funcionais lineares.
        \item Temos
        %
        \begin{align*}
            \xi'(x)\cdot h &= \lim_{t\to 0} \frac{\xi(x+th) - \xi(x)}{t} \\
                           &= \lim_{t\to 0} \frac{1}{t}\cdot\left[ \langle f(x+th), g(x+th) \rangle 
                           - \langle f(x), g(x) \rangle \right] \\
                           &= \lim_{t\to 0} \frac{1}{t}\left[ \langle f(x+th), g(x+th) \rangle
                           - \langle f(x), g(x+th) \rangle
                           + \langle f(x), g(x+th) \rangle 
                           - \langle f(x), g(x) \rangle \right] \\
                           &= \lim_{t\to 0} \left\langle \frac{f(x+th) - f(x)}{t}, g(x+th) \right\rangle
                           + \left\langle f(x), \frac{g(x+th) - g(x)}{t} \right\rangle \\
                           &= \langle f'(x)\cdot h, g(x) \rangle + \langle f(x), g'(x)\cdot h \rangle
        \end{align*}
        %
        novamente usando o fato de que $f$ e $g$ são funcionais lineares e propriedades do produto interno
        (inclusive que ele é contínuo).
    \end{enumerate}
%
\paragraph{Exercício 2.}
    a
%
\paragraph{Exercício 3.}
    a
%
\paragraph{Exercício 4.}
    a
%
\paragraph{Exercício 5.}
    a
%
\paragraph{Exercício 6.}
    a
%
\paragraph{Exercício 7.}
    a
%
\end{document}